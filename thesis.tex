%% LaTeX2e class for student theses
%% thesis.tex
%% 
%% Based on SDQ KIT Template by Erik Burger
%%
%% Karlsruhe Institute of Technology
%% Institute for Automation and Applied Informatics
%% AIDA Research Group
%%
%% Nicole Ludwig
%% nicole.ludwig@kit.edu
%%
%% Version 1.2, 2018-10-11

%% Available page modes: oneside, twoside
%% Available languages: english, ngerman
%% Available modes: draft, final

\documentclass[oneside,english]{thesisIAI}

%% ---------------------------------
%% | Additonal Packages |
%% ---------------------------------
\usepackage{amsmath}
\usepackage{amsfonts}
\usepackage{amssymb}
\usepackage{pdfpages}
\usepackage{booktabs}
\usepackage{siunitx}
\usepackage[utf8]{inputenc}
\usepackage{footnote}
\usepackage{tocloft}
\usepackage{hyperref}

\usepackage{graphicx}
\usepackage{epstopdf}

\usepackage{colortbl}
\definecolor{gray90}{gray}{0.9}
\usepackage{float}

\usepackage{transparent}
\usepackage{color}
\graphicspath{{figure/}}



\usepackage{tocbasic}
\usepackage{caption}

%%% Doc: ftp://tug.ctan.org/pub/tex-archive/macros/latex/contrib/caption/caption.pdf
\usepackage{caption}
% Aussehen der Captions
\captionsetup{
   margin = 10pt,
   font = {rm},
   labelfont = {bf},
   format = plain, % oder 'hang'
   indention = 0em,  % Einruecken der Beschriftung
   labelsep = space, %period, space, quad, newline
   justification = RaggedRight, % justified, centering
   singlelinecheck = true, % false (true=bei einer Zeile immer zentrieren)
   position = bottom %top
}


%%% Bugfix Workaround
\DeclareCaptionOption{parskip}[]{}
\DeclareCaptionOption{parindent}[]{}
\DeclareUnicodeCharacter

\usepackage{csquotes}
\usepackage{cleveref}
\usepackage{siunitx}

%% useful abreviations

\newcommand\ie{i.\,e.\xspace}
\newcommand\eg{e.\,g.\xspace}
\newcommand\Eg{E.\,g.\xspace}
\newcommand\NB{N.\,B.\xspace}
\newcommand\BSc{B.\,Sc.\xspace}
\newcommand\MSc{M.\,Sc.\xspace}
\newcommand\PhD{Ph.\,D.\xspace}
\newcommand\etc{etc.\xspace}
\newcommand\cf{cf.\xspace}
\newcommand\Cf{Cf.\xspace}
\newcommand\etal{et\,al.\xspace}
\newcommand\page[1]{p.\,#1}
\newcommand\pages[1]{pp.\,#1}
\newcommand\ham{a.\,m.\xspace}
\newcommand\hpm{p.\,m.\xspace}
	
\newcommand\zB{z.\,B.\xspace}
\newcommand\proz{\,\%\xspace}

%% von mir hinzugefügt




%% ---------------------------------
%% | Information about the thesis  |
%% ---------------------------------

%% Name of the author
\author{Vivien Geenen}

%% Title (and possibly subtitle) of the thesis
\title{Using a Building as Thermal Storage and Model Predictive Control of a Heat Pump for Grid Stabilization}

%% Type of the thesis 
\thesistype{Master's thesis}
%\thesistype{Seminar Paper}

%% Change here the faculty 
%\faculty{Seminar}
\faculty{Engineering}
%\faculty{Informatics}

%% The advisors are PhDs or Postdocs
\advisorone{Moritz Frahm, M.Sc. and Frederik Zahn, M.Sc}

%% The second advisor can be omitted
%\advisorone{}

%% Please enter the start end end time of your thesis
\editingtime{14. June}{14. December 2021}

\settitle

%% Please do not change anything in this tex file without talking to your supervisor
\input{tex/formats}

%% --------------------------------
%% | Settings for word separation |
%% --------------------------------

%% Describe separation hints here.
%% For more details, see 
%% http://en.wikibooks.org/wiki/LaTeX/Text_Formatting#Hyphenation
\hyphenation{
% me-ta-mo-del
}

%% --------------------------------
%% | Bibliography                 |
%% --------------------------------

% Please make sure your bibfiles name is thesis.bib and is located in the tex subfolder
\input{tex/bibliography} 

%% Set PDF metadata, must be set BEFORE \begin{document}
\setpdf

%% ====================================
%% ====================================
%% ||                                ||
%% || Beginning of the main document ||
%% ||                                ||
%% ====================================
%% ====================================
\begin{document}

\hypersetup{pageanchor=false}

%% Set the title
\maketitle

%% The Preamble begins here
\frontmatter

\input{sections/declaration.tex}

\setcounter{page}{1}
\pagenumbering{roman}

%% ----------------
%% |   Abstract   |
%% ----------------
 
%% For theses written in English, an abstract both in English
%% and German is mandatory.
%%
%% Seminar Papers written in English need only an English abstract.
%%
%% For theses written in German, a German abstract is sufficient.
%%
%% The text is included from the following files:
%% - sections/abstract
%% - sections/zusammenfassung
\chapter{Kurzfassung}
\label{ch:kurzfassung}

\pdfbookmark[1]{Abstract}{Abstract}

\chapter{Abstract}

The increasing of renewable energy sources into the grid leads to more electricity fluctuations in the power generation. Demand side management (DSM) can be an instrument to compensate for the fluctuations. The use of DSM for heat pumps in residential buildings has potential. However, the challenge is to achieve user comfort via thermal storage in a water reservoir or the thermal capacity of the building itself. Model predictive control (MPC) is a suitable tool to realize DSM by integrating forecasts.\newline
The thesis presents a simulative investigation of an MPC of a heat pump for a real-world building from the Energy Lab 2.0 at the KIT. A thermal building model is created with measurements from the reference building applying the grey-box approach. To record the training data and the verification data, the building is heated with electrical heaters in two experiments.\newline
An MPC is generated concerning comfort and grid services requirements in the cost function. The comfort requirement depends on the occupancy schedule of the reference building. In the case of the presence of occupants, a desired comfortable temperature is dictated, while in case of absence only the grid service is of interest. This MPC configuration is called the basic scenario. Further scenarios are investigated to evaluate the influence of an occupancy schedule on comfort and grid services. A concluding discussion shows that the occupancy schedule impacts positively the comfort requirement with less energy consumption and low costs for grid services when using the basic scenario.

\newpage

\hypersetup{pageanchor=true}
%% ------------------------
%% |   Table of Contents  |
%% ------------------------
\tableofcontents
\listoffigures
\listoftables
\setcounter{secnumdepth}{5}
\setcounter{tocdepth}{5}

%% -----------------
%% |   Main part   |
%% -----------------

\mainmatter

\chapter{Introduction}
\label{ch:introduction}
Climate change is challenging the entire world. In the Paris Agreement, the United Nations (UN) \nomenclature[A]{UN}{United Nations} agrees to keep the rise in global average temperature significant under two degrees Celsius \cite{UnitedNations.2015}. To achieve this aim every nation has to reduce its greenhouse gas emissions. This calls for changes in the mobility sector, industry, and energy production, for example. The German government intends to implement this by promoting electromobility, using hydrogen in industry, and energy transition towards renewable energy sources \cite{Deutschlandfunk.24.06.2021}. In particular, the so-called energy transition that has already been initiated has to be driven forward. That means the expansion of renewable energies and decreasing conventional power plants. The German government is aiming to phase out coal-fired power plants by 2038 \cite{bundesregierung.2021}. For covering the energy demand, a high increase in energy from renewable sources, e.g. photovoltaics and wind power, is necessary for the coming years. 
\newline
Unfortunately, a disadvantage of these renewable energy sources is that they fluctuate with the weather and cannot produce energy on demand. In addition, more renewable energy sources can lead to more intense imbalances in the grid.
To compensate the imbalances in the grid, the grid requires or releases energy. In order to have enough energy available, the amount of storage options must grow with the amount of renewable energy sources. Batteries, pumped hydroelectric energy storage, thermal energy storage, and many other technologies could store an excess of power during a sunny or windy day. Furthermore, the necessity of demand side management (DSM) \nomenclature[A]{DSM}{Demand Side Management} rice also with the increasing renewable energy sources, then DSM can shift loads to balance the grid. Load shifting is part of DSM \cite{Gellings.1985} and already used in the industry. Another approach is to use residential buildings as thermal storage and demand response to contribute to grid balance\cite{Kohlhepp.2017}. As a promising DSM technology, the control of heating, ventilation, and air conditioning (HVAC) systems could be used. In particular, controlling heat pumps of buildings seem auspicious. As at least 1.25 million heat pumps are already installed in Germany, and their number is increasing steadily \cite{BMW.2021}.
\newline
The implementation of this approach needs a control strategy ensuring consumer comfort also during changing weather conditions. Model predictive control (MPC) \nomenclature[A]{MPC}{Model Predictive Control} is a suitable tool to integrate forecasts of weather and control heat pumps in buildings for stabilizing the grid with the thermal storage of the building. Research has already shown the possibilities of MPC to shift loads, to save energy and costs \cite{Oldewurtel.2010}, \cite{Hazyuk.2012b}, \cite{Zwickel.2019}.
On the other hand, researchers investigate the impact of occupancy plans on energy consumption in buildings. They prove a significant energy-saving potential \cite{Wang.2019}. This thesis picks up the advantages of an occupancy plan, and it analyses consumption, comfort, and grid service of an MPC with and without an occupancy plan of the building.
 

\section{Objective of this work}
\label{section:obejective}
    This thesis aims to design a control system, which simultaneously serves the grid and complies with the required comfortable internal temperature range, for the heat pump of a building in the so-called "Living Lab" of the Karlsruhe Institute of Technology (KIT) \nomenclature[A]{KIT}{Karlsruhe Institute of Technology} at Campus North. The implementation is to be carried out using the control method Model Predictive Control. This method enables to predict the future thermal behaviour of the building and to react to the current and future fluctuations of the weather or the grid for example. 
    In the first step, a thermal model of the building behaviour must be created. For this purpose, the physical structure of the model is to be determined. Appropriate assumptions can be made to reduce the complexity of the thermal behaviour of the building. Furthermore, the parameters of the model are be determined by a parameter identification from measurements to obtain a grey-box model, i.e., a combination of physical model structure and optimisation with measurement data. After the verification of the thermal model using measured data from the Energy Lab 2.0 from the KIT, an optimal control problem is be created. The aim is to construct an MPC algorithm and to simulate its application. The software used will be Matlab/Simulink.

\section{Related work}
\label{section:relatedwork}

    Extant literature investigates thermal modelling and controlling of buildings. Kramer et al. \cite{Kramer.2012} summarize in a literature review thermal modelling approaches such as white-box, grey-box, and black-box models and present how researchers apply these approaches. Further authors identify their thermal model parameters with measurements and use the grey-box modelling approach \cite{Harb.2016}, \cite{Freund.2020}, \cite{EvelynSperber.2019}. Coakley et al. \cite{Coakley.2014} see the advantages of grey-box modelling in the short development time for the model, fidelity of predictions, and the interaction of building, system and environmental parameters. One disadvantage is that modellers need a high level of knowledge in physical and statistical modelling \cite{Coakley.2014}. Furthermore, Cigler et al. \cite{JiriCigler.} and Hazyuk et al. \cite{Hazyuk.2012b} see the advantages of grey-box models and work with them in their MPC applications for thermal management in buildings. 
    \newline
    Regardless of the type of model, MPC is utilised for control of heating, ventilation, and air conditioning (HVAC) \nomenclature[A]{HVAC}{Heating, Ventilation, and Air Conditioning} systems in buildings for a variety of reasons, e.g. \cite{Hazyuk.2012b}, \cite{Zwickel.2019}, \cite{Oldewurtel.2010}. Researchers are interested in the reduction of energy consumption \cite{Hazyuk.2012b} and saving costs \cite{Zwickel.2019} while obtaining thermal comfort. Oldewurtel et al. \cite{Oldewurtel.2010} present how to decrease or shift the peak load of buildings.
    \newline
    On the other hand, \cite{WPimBestand.2020} and \cite{Kohlhepp.2017} refer to the potential of heat pumps for grid services. Among others, the report "Wärmepumpen in Bestandsgebäuden" examines the load shifting potential of grouped heat pumps. The researchers determine 4 to 14 GWh load shifting potential for one million heat pumps \cite{WPimBestand.2020}.
    Kohlhepp and Hagenmeyer \cite{Kohlhepp.2017} introduce a method to assess the technical potential of HVAC systems for grid services. Especially, they detect 5.2 TWh electrical demand for heat pumps in Germany per year.
    \newline
    Avci et al. \cite{Avci.2013} apply MPC for heat pumps to the realization of grid services. They give an early indication of the potential of grid services using real-time pricing. For application of DSM, most researches apply a dynamic price signal, although their focus differs e.g. according to the type of buildings \cite{Bianchini.2019}, \cite{Kim.2018} or the type of optimisation \cite{Bianchini.2019}, \cite{Bianchini.2016}.\newline 
    Another interesting field of research is the energy saving potential by planing the occupancy of buildings. Wang et al. \cite{Wang.2019} show in their paper that 13 percent of energy can be saved by occupancy-based controls for an office building. Liang et al. \cite{Liang.2016} investigate an occupancy schedule with machine learning approaches to better control HVAC systems and consequently to save energy.\newline
    
    This thesis integrates occupancy schedules into the MPC formulation to investigate the potential of included occupancy behaviour on control metrics. A simple occupancy plan is used to examine whether comfort, grid services, and energy consumption can be improved compared to an MPC without an occupancy plan. Consequently, in this thesis, an MPC is created with a grey-box model and grid services are implemented with real-time pricing, similar to the papers cited above. But in this thesis, the potential of an occupants schedule is analysed at a real reference building during focusing on the aim of grid services. This thesis finds an answer to the question: How does the inclusion of an occupancy schedule in an MPC improve grid services, energy consumption and comfort? 
 %the interaction of building, system and environmental parameters vllt umbenennen, weil er es auch so nennt!!

\section{Content structuring}
\label{section:contentstructuring}
To answer the research question a thermal model is first created with help of an experiment. After that, the model can be integrated into the constructed MPC. The MPC implementation can then be used to generate the results of different scenarios.
Chapter 2 concerns foundations, which are required to solve the objective of this work. There are represented some thermal and modelling basics and an introduction in an MPC.
Chapter 3 explains the modelling approaches white-box and grey-box modelling with some advantages and disadvantages. Further, the submodels water reservoir and building model are introduced. The parameter identification is explained especially for the building model, and the verification results are shown. At last, the completed model is written in the state-space formulation.
Chapter 4 explains the structure of the experiments for the verification data set and the training data set.
Chapter 5 describes the MPC approach for a basic scenario. What kind of data, the constraints and the cost function are explained in detail. Furthermore, the workflow of the MPC is pictured, and the choice of the predictive horizon is performed. Variables are introduced concerning the comfort and grid service requirement that we use for evaluating the results.
Chapter 6 introduces the scenarios at first. After that, the results are depicted, and at last, the explanation of the results follows in the discussion.
Chapter 7 summarise this thesis and gives an outlook about future steps.
\chapter{Foundations}
\label{ch:foundations}

This work is based on foundations, which are summarized in this chapter. This includes thermal basics, foundations about thermal modelling, and model predictive control (MPC). 
%Note, that in the following all vectors are denoted as bold letters, e.g. \textbf{x}.

\section{Thermal basics}
\label{section:thermalbasics}

\subsection{Balancing energy}
\label{subsection:balancing energy}

    It is necessary to comprehend the basics of thermodynamics to understand the structure of a thermal model. The first law of thermodynamics is the general energy balance and is formulated for unsteady and open systems as follows \cite{Baehr.2016b}:
    \begin{equation}
        \label{eq:1HS}
        \sum_i \dot{Q_\text{i}} + \sum_j \dot{W_\text{j}} + \sum_k \dot{m_\text{k}}*(h + \frac{c^2}{2} + gz)_\text{k} = \frac{d}{dt} \sum_l U_\text{l}
    \end{equation}
    
    In terms of a building, we set the work $\dot{W}$ to zero according to the relationship $W =\int Pdt - \int pdV$ \cite{Baehr.2016b} because a building can't change the volume $V$, and we have no additional mechanical power $P$. If we have no mass flow $\dot{m}$ in our system, we obtain a closed system. Regarding buildings, mass flows could be airflow through the window, for example. Then we also consider the enthalpy $h$, the fluid velocity $c$, the high $z$ and the gravitational acceleration $g$.
    \newline
    Since we do not consider airflow, we use the closed system with the heat flows $\dot{Q_\text{i}}$ and the inner energy $U_\text{l}$.
     \begin{equation}
        \label{eq:1HS}
        \sum_i \dot{Q_\text{i}}  = \frac{d}{dt} \sum_l U_\text{l}
    \end{equation}
    The deduction of the inner energy $U$ starts with the complete differential description of the specific inner energy $du$ as \cite{Baehr.2016b}:
    \begin {equation} 
    \label{eq:innerEnergy}
    du = (\frac{\partial u}{\partial T})_\text{v} dT + (\frac{\partial u}{\partial v})_\text{T} dv
    \end{equation}
    The specific volume $dv$ is negligible in buildings, and the specific heat capacity during constant volume has the expression $c_\text{v} = (\frac{\partial u}{\partial T})_\text{v}$ \cite{Baehr.2016b}. After integrating and replacing the specific values by volume, we obtain the relation for the inner energy $U$, with the mass $m$ and the temperature difference $\delta T$. 
        \begin{equation}
        \label{eq:innerEnergy}
        U = m c_\text{v} \Delta T
    \end{equation}
    
    It applies to substances with a specific volume regardless of the pressure that $c_\text{v}=c_\text{p}=c$. 
    We sum the heat flows in the energy balance in \autoref{eq:1HS}. However, what kind of heat transfer is there, the following few chapters explain.
    
    \vspace{2cm}
    
    There are three mechanisms of heat transfer: Heat conduction, heat convection, and heat radiation \cite{.2013}. Thermal modelling of buildings requires all of these mechanisms. For example, conduction is the primary part of heat transfer through walls or floors. Convection occurs on the inside and the outside of the building between the walls and the air. To integrated the impact of the sun, radiation is needed, for example.

\subsection{Conduction}
\label{subsection:conduction}

    Conduction means that heat energy is directed in a solid or fluid. Molecules within the solid or fluid have higher energy when the temperature is higher. They transfer the energy to neighbouring molecules with smaller energy. Without a heat source, the temperature difference between a hot and a cold location of the molecules decreases.\cite{Kuchling.2007}
    \newline The equation
    \begin{equation}
    \label{eq:fourier}
        \dot{\textbf{q}} = - \lambda \nabla T
    \end{equation}
    describes the conduction according to Fourier \cite{.2013}. There is $\lambda$ the thermal conductivity with the assumption of being constant and $\dot{\textbf{q}}$ and $T$ represent the specific heat flux and the temperature. The thermal conductivity is dependent on the material, such as concrete, wood or bricks. 
    \newline
    To know the heat flux $\dot{Q}$, it is necessary to expand the above equation with the area $A$, the thickness of the conductive medium $d$ and a temperature difference $\Delta T$ assuming one significant direction of the heat flux $\dot{Q}$ to:
    \begin{equation}
    \label{eq:conduction1}
        \dot{Q} = \frac{A\lambda}{d} \Delta T
    \end{equation}
    In terms of buildings, the conductive medium could be walls, floors or roofs.

\subsection{Convection}
\label{subsection:convection}

    Macroscopic movements of a fluid lead to the transport of kinetic energy and enthalpy. This mechanism is called convection. These movements are generated by external forces or by internal forces like balancing the pressure or temperature.\cite{.2013}
    \newline
    Newton's law of cooling describes the convective heat transfer $\dot{Q}$ as 
    \begin{equation}
    \label{eq:newton}
        \dot{Q} = \alpha A (T_w - T_\infty)
    \end{equation}
    with the heat transfer coefficient $\alpha$, especially for building modelling the wall temperature $T_w$ and the environment temperature $T_\infty$ \cite{Griesinger.2019}
    . There are two possibilities to determine the heat transfer coefficient. Both require a temperature difference $\Delta T$ and either a temperature gradient $\partial T/\partial x$ or a heat flux $\dot{Q}$.
    \cite{.2013} 

\subsection{Radiation}
\label{subsection:radiation}

    Every body emits heat radiation to the environment with electromagnetic waves. Especially, heat radiation does not need matter for energy transportation. As shown in the following equation, the temperature $T$ of the body influences heat radiation.\cite{.2013} 
    \begin{equation}
    \label{eq:radiation}
        \dot{q} = \sigma T^4
    \end{equation}
    This correlation applies to a black body, where $\dot{q}$ is a heat flux and $\sigma$ represents the Stefan- Boltzmann coefficient. A black body absorbs all heat radiation with all wavelengths from all directions\cite{Griesinger.2019}. The consideration of a black body is idealized. For the illustration of a real body (see \autoref{eq:realbodyradiation}), the emissivity $\epsilon$ is used. $\epsilon$ is material-dependent and lies between 0 and 1.
    \begin{equation}
    \label{eq:realbodyradiation}
        \dot{q} = \epsilon \sigma T^4
    \end{equation}
    In general, a body absorbs, transmits, and reflects radiation with the appropriate coefficients $a$, $\tau$ and $r$. The sum of three coefficients has to be one ($a + \tau + r = 1)$
    \cite{Baehr.2016}.
    \newline
    The primary source of heat radiation is the sun, which plays an important role in the thermal modelling of buildings. Objectives in the building, such as radiators, also radiate heat. For example, radiators have equal parts convective and radiative energy transport \cite{Hazyuk.2012}. 
    
\section{Lumped capacitance model}
\label{section:modelling}
For modelling the thermal behaviour of buildings, the lumped capacitance model is often used. With this approach, using the electrical analogy, building elements are represented by resistors $R$ and capacitors $C$. \cite{Kramer.2012}

\subsection{Electrical analogy}
\label{electricalanalogy}

    Similar to an electrical network, the potential is represented by the temperature at one node and the heat flux corresponds to the current. We can also use Ohm's law, which is formulated in a thermal way as:  
     \begin{equation}
    \label{eq:Ohm}
        \dot{Q} = \frac{\Delta T}{R} 
    \end{equation}
    Combining the above equation with \autoref{eq:conduction1} or \autoref{eq:newton}, the thermal resistance $R$ is determined in conductive cases as \cite{Kuchling.2007}:
    \begin{equation}
    \label{eq:r_lambda}
        R_\lambda = \frac{d}{A\lambda}
    \end{equation}
   and in convective cases as\cite{Griesinger.2019}:
    \begin{equation}
        R_\alpha = \frac{1}{\alpha A}
    \end{equation}
    
    \begin{figure}[h]
   %\captionof{figure}[]{}
    \begin{minipage}[t]{7cm}
    \vspace{0pt}
    Thermal resistances can sum up to one thermal resistance, even if they are from different mechanisms of heat transfer. Based on an example in \autoref{fig:sampleWall}, the addition is explained. The figure shows a section of a wall with a heat flow $\dot{Q}$, the ambient temperature $T_\text{1}$ and $T_\text{2}$ separated by that wall. We have three thermal resistances $R_{\alpha\text{,1}}$, $R_{\alpha\text{,2}}$, and $R_\lambda$, which we sum to one thermal resistance $R = R_{\alpha\text{,1}} + R_\lambda + R_{\alpha\text{,2}}$. Now, we can calculate the heat flow $\dot{Q}= \frac{T_\text{2}-T_\text{1}}{R} $ according to \autoref{eq:Ohm}.
    \end{minipage}
    \hfill
    \begin{minipage}[t]{7cm}
    \vspace{0pt}
    \centering
    \input{figure/Wand um Widerstände zu addieren.pdf_tex}
    \caption{Sample of a wall with thermal resistances}
    \label{fig:sampleWall}
    \end{minipage}
    \end{figure}
    
    In sum, the thermal resistances $R$ comply with electrical resistors.
    Further for modelling thermal networks, the thermal capacitance $C$ is needed. It is calculated from the specific heat capacity $c$ multiplied by the mass $m$ ($C=cm$).
    \newline
    For a better explanation of the structure of a thermal network, a simple example is depicted in \autoref{fig:sampleRCnetwork}. It represents a heated wall of a building.
    \begin{figure}[h]
    \centering
    \def\svgwidth{300pt}
    \input{figure/beispiel Netzwerk.pdf_tex}
    \caption{Sample RC- network}
    \label{fig:sampleRCnetwork}
    \end{figure}
    The heat flux $\dot{Q}$, for example from a radiator, influences the temperature $T_\text{wall}$, as well as the capacitance $C$. And the temperature $T_\text{wall}$ affects the temperature inside and outside $T_\text{inside}$ and $T_\text{outside}$ with their resistances $R_{\alpha \text{,inside}}$ and $R_{\alpha \text{,outside}}$. The example shows that all connections in the network influence each other.
    To model the dynamics of the wall in differential equations, Kirchhoff's Current Law is required. It states that the sum of the flowing current to the node is equal to the sum of the flowing current of the node 
    \cite{Kuchling.2007}. 
    Because of the thermal analogy of electrical laws, the current is replaced by heat flux. The following differential equation results for the node  $T_\text{wall}$ using Ohm's law ($\dot{Q}=\Delta T/R$) and the relationship $\dot{Q}=C\frac{\partial T}{\partial t}$.     
    \begin{equation}
    \label{eq:sampledifferential}
    C \frac{\partial T_\text{wall}}{\partial t} = \dot{Q} + \frac{T_\text{inside}-T_\text{wall}}{R_{\alpha \text{,inside}}} - \frac{T_\text{wall}-T_\text{outside}}{R_{\alpha\text{,outside}}}
    \end{equation}
    In \autoref{fig:sampleRCnetwork}, the thermal resistances are serially connected. According to the electrical network, resistances in series are equal to their sum. 
    \begin{equation}
    \label{eq:resistanceseriel}
        R_\text{sum} = R_{\alpha \text{,inside}} + R_{\alpha \text{,outside}}
    \end{equation}
    A parallel circuitry has windows and walls in buildings, for example. Here the resistances are calculated according to the following schema:
    \begin{equation}
    \label{eq:resistancesparallel}
        \frac{1}{R_\text{sum}} = \frac{1}{R_\text{wall}} + \frac{1}{R_\text{window}}
    \end{equation}
    In terms of needed more capacitances for describing the thermal model, the summary capacitance is added in a parallel circuitry as: 
    \begin{equation}
    \label{eq:capacityparallel}
        C_\text{sum} = \sum \limits_1^i C_\text{i} 
    \end{equation}
    The serial circuitry of capacitances is calculated as follows:
    \begin{equation}
    \label{eq:capacityseriell}
       \frac{1}{C_\text{sum}} = \sum \limits_1^i \frac{1}{C_i} 
    \end{equation}
    
%maschenregel noch erklären?


\section{Model predictive control (MPC)}
\label{section:mpc}

Model predictive control exploits models of the plant to predict and optimise the behaviour of the plant \cite{Grune.2017}.
Applied to thermal control of a building with the aim of grid-supporting, a model of the thermal behaviour of the building is required to predict the reaction of the system behaviour in the next $N$ time steps, called the prediction horizon. Every time step $k$, the current state $\mathbf{x_k}$, the output $\mathbf{y_k}$ is measured, and the future system behaviour is obtained by computation. The computation of the future system behaviour may include measurable disturbances such as weather forecast, occupancy schedule and the optimisation of the control signal $\mathbf{u_k}$ over the optimisation horizon $\mathbf{u_{k+N}}$. However, only the first calculated control signal is adopted as input for the plant.
Then, the calculations are repeated at every time step. \autoref{fig:sampleMPC} visualises the MPC control loop.
 \begin{figure}[h]
    \centering
   \def\svgwidth{320pt}
    \input{figure/MPC beispiel.pdf_tex}
    \caption{MPC structure of the control loop}
    \label{fig:sampleMPC}
    \end{figure}
\newline
Concluded, the MPC is "an iterative online optimisation over the predictions"
\cite{Grune.2017} 
compiled by the thermal model of the building. Mathematically explained, the optimizer needs to minimize the following equation according to
\cite{Kouvaritakis.2018}
and
\cite{Oldewurtel.2012}:
\begin{align}
\label{eq:costfunc}
\textrm{Cost function} && \text{minimize} \sum_{k=1}^{N-1} c_\text{k}(\mathbf{x_k,u_k,y_k})
\end{align}
subject to 
\begin{align*}
\textrm{Current state} && \mathbf{x_0} &=& \mathbf{x} \\	
\textrm{Dynamics} && \mathbf{x_{k+1}}&=& f(\mathbf{x_k,u_k,d_k})		&&	\mathbf{y_k} = g(\mathbf{x_k,u_k,d_k})\\				
\textrm{Constraints} && \mathbf{y_{min}}&\leq& \mathbf{y_k} \leq \mathbf{y_{max}}\\		
\textrm{} && \mathbf{u_{min}}&\leq& \mathbf{u_k} \leq \mathbf{u_{max}}	
\end{align*}
$c_\text{k}$ represents the cost function, which is explained in detail in \autoref{subsection:costfunction}
. In terms of building control, $y$ is the internal temperature.
%Störungen noch erklären

\subsection{Cost function}
\label{subsection:costfunction}

    Generally, the cost function $c_\text{k}$ assigns a cost to the control signal $\mathbf{u_k}$ and the current state $\mathbf{x_k}$, which is mathematically described in
    \autoref{eq:costfunc}
    , with:
    \begin{equation}
    \label{eq:c_k}
    c_\text{k} = (\mathbf{x_k^T}Q\mathbf{x_k}+\mathbf{u_k^T}R\mathbf{u_k})
    \end{equation}
    Here $Q$ and $R$ are matrices over which individual elements of the state vector or control signal vector can be weighted differently.  
    \cite{Kouvaritakis.2016}
    Especially for every application, the cost function has an individual form to reach the aims of the MPC.
    
\subsection{Dynamics}
\label{subsection:dynamics}
    
    The state-space formulation (SSF) is an alternative representation of a linear differential equation, which models a physical system. In this work, it is used for the formulation of the thermal model, which is required for the MPC. The SSF consists of the state $\textbf{x}$, the control signal $\textbf{u}$, the disturbances $\textbf{d}$ and the output of the system $\textbf{y}$ are represented in \autoref{eq:statespace}. The system matrix is $A$, $B_\text{1}$ and $B_\text{2}$ are called the input matrices, $C$ is the output matrix, $D_\text{1}$ and $D_\text{2}$ are the pass-through matrices. The \autoref{tab:matrixDim} lists the dimensions of the matrices m x n with m rows and n columns.   
    \begin{align}
    \label{eq:statespace}
    \dot{\textbf{x}}=A\textbf{x}+B_1\textbf{u}+B_2\textbf{d}\\
    \textbf{y}=C\textbf{x}+D_1\textbf{u}+D_2\textbf{d} \notag
    \end{align}
    
    \begin{table}[]
        \centering
        \begin{tabular}{c|c|c}
            & $m$ & $n$  \\
            \hline
            $A$ & number of states & number of states\\
            $B_\text{1}$ & number of states & number of control signals\\
            $B_\text{2}$ & number of states & number of disturbances\\
            $C$ & number of outputs & number of states\\
            $D_\text{1}$ & number of outputs & number of control signals\\
            $D_\text{2}$ & number of outputs & number of disturbances\\
        \end{tabular}
        \caption{dimensions of the matrices}
        \label{tab:matrixDim}
    \end{table}
    Every differential equation needs initial values for solving. Therefore, initial states $\mathbf{x_0}$, initial control signals $\mathbf{u_0}$, and initial disturbances $\mathbf{d_0}$ must be given.
    In a thermal model of a building, some authors (\cite{Hazyuk.2012}, \cite{Siroky.2011}) use the state as a vector of some temperatures, the control signal as a signal for the heating system, the disturbances can describe the influence by the weather or occupants and the output of the system contains frequently the temperature inside of the building.
    


\subsection{Constraints}
\label{subsection:constraints}

Dealing with constraints is one of the most important advantages of MPC. Thereby, constraints can be used for the state, the output, and the input. In terms of building control, output constraints and input constraints are reasonable, as mathematically described in the \autoref{eq:costfunc}. That means, the output constraints could be a temperature range, which feels comfortable for occupants. And the constraints for the input are given as minimal (= 0) and maximal values of the possible performances. General, logical and physical ranges are constrained. There are different forms of constraints, but linear constraints are frequently used for MPC because they simplify the optimisation problem. Constraints can also be time dependant. This is beneficial for embedding diverse temperature ranges during the night and the day or during the working time of occupants when they are not at home.
\cite{Siroky.2011}
%es gibt harte und weiche constraints. wichtig ist, dass man nicht auf beiden seiten harte constraints einstellt. konvexe Constraints bei bedarf

\section{The reference building}
\label{section:building}
    Since this thesis is based on a real building, some necessary details about the building are described below.
    \newline
    The building is located on the "Campus Nord" of the "Karlsruher Institut für Technologie (KIT)" and is part of the "Energy Lab 2.0", "a research infrastructure for renewable energy"\cite{KIT.2021}. It is equipped with a kitchen, a bathroom, five rooms and a technical room. For a better orientation, \autoref{fig:Bauplan} shows a part of the construction plan of the building. The building is designed as a single-family house, but for practical reasons, it is used as an office. The living area is around 100 $m^2$. The building offers two options to heat or cool with a ground-source heat pump or an air heat pump. The focus is on the air heat pump because the most commonly used heat pumps in Germany are air heat pumps \cite{bwp.2021}. In addition to the heat pump, there is a water reservoir for saving energy with stratified storage. The total volume is 1000 litres \cite{Oskar}. The heating system inside the building is provided as underground floor heating. However, the heating system is not completely installed yet. So using the heat pump, the water reservoir or the underground floor heating is not possible actual.
    \newline
    One of the main features of the building is the number of sensors. The air temperature is measured in every room, as well as the temperature in the middle of the exterior wall, the screed temperature, the floor plate temperature, and the temperature of the inner wall between room three and room four (see \autoref{fig:Bauplan}). Furthermore, the consumption of the actual electrical power is also detected. Only the mentioned sensors are needed in this case, but there are many more sensors.
    %Himmelsrichtung noch beschreiben
    \begin{figure}
        \centering
        \def\svgwidth{320pt}
        \input{figure/Bauplan_WärmepumpenhausAuschnitt.pdf_tex}
        \caption{Construction plan of the building \cite{Bauplan}}
        \label{fig:Bauplan}
    \end{figure}
\chapter{Modelling}
\label{ch:modelling}
After explaining thermal basics and the electrical analogy, the foundations are used in this chapter. How I obtain a thermal model and the resulting thermal model, is introduced. Later the model is required for the MPC to predict the thermal reactions of the building. 

\section{The building}
\label{section:building}
Because of Da sich diese ganze arbeit auf ein reales Gebäude bezieht, wird im folgenden das Gebäude näher beschrieben.
\newline
The building stands at the "Campus Nord" from the "Karlsruher Institut für Technologie (KIT)" and is part of the "Energy Lab 2.0", "a research infrastructure for renewable energy"\cite{KIT.2021}. It is equipped with a kitchen, a bath room, five rooms and a technical room over with. In \autoref{fig:Bauplan}, you can see a part of the construction plan of the building

\begin{figure}
    \centering
    \def\svgwidth{320pt}
    \input{figure/Bauplan_WärmepumpenhausAuschnitt.pdf_tex}
    \caption{construction plan of the building \cite{Bauplan}}
    \label{fig:Bauplan}
\end{figure}
\section{The thermal model}
\label{thermalmodel}

\section{Data collection}

\section{Validation of the thermal Model}
\label{validationthermalmodel}




\chapter{Experiment}
\label{ch:experiment}
Industrial heater (IH)
Household heater with fan (HHF)
Household heater wtihout fan (HH) 
\section{}
\label{}
\section{}
\label{}
\section{}
\label{}

\chapter{Model predictive control}
\label{ch:mpc}
In this chapter, the framework conditions of the MPC are described by answering the questions: (i) what is controlled? (ii) How is controlled? (iii) Which curves are controlled? What data are used? Also, the constraints, the cost function, and the workflow of the MPC script are introduced.  

\section{Framework conditions of the MPC}
\label{section:FrameworkMPC}

\section{Constrains}
\label{section:constrains}
\section{Cost function}
\label{section:costfunction}
\begin{equation}
    \text{minimize} \sum_{k=1}^{N-1} w_\text{1}\cdot (y-y_\text{track})^2 + w_\textbf{2}\cdot(u_\text{2}\cdot grid)^2 + w_\text{3} \cdot \eta^2
\end{equation}
\section{Workflow of the MPC script}
\label{section:costfunction}
\begin{figure}[h]
            \centering
            \def\svgwidth{0.6\textwidth}
            \input{figure/workflowMPC.pdf_tex}
            \caption{Workflow of the MPC script}
            \label{fig:workflowMPC}
    \end{figure}
\chapter{Results and discussion}
\label{ch:results}
In this chapter, different scenarios are considered to answer the research question focusing on grid services. Several scenarios are supposed to explain how the occupancy schedule affects energy consumption, grid service and comfort: (i) a scenario with an occupancy schedule, in which the MPC has to comply with fewer restrictions during the absence of occupants; (ii) a scenario that also has a lower temperature specification during the absence; and (iii) a scenario without an occupancy schedule. In the last section, the results are evaluated, and an answer to the research question is found.

\section{Results of the scenarios}
\label{sec:ResultsScenarios}
In this section, the implementation of the different scenarios is explained and the results of the temperature and control signal curves are presented. Furthermore, energy consumption, comfort and grid services are evaluated.  

\subsection{Presentation of the scenarios}
\label{subsec:Presentation of the scenarios}

\textbf{(i) The basic scenario:}\newline
The first considered scenario is the basic scenario, which is explained in detail in \autoref{ch:mpc}. Summarised, we desire the $T_\text{inside}$ during the presence of occupants as 22°C. And we leave the optimiser free to find an optimal temperature in the specified temperature range during the absence realised by neglecting the comfort requirement in the cost function.\newline

\textbf{(ii) Scenario 2:}\newline
The Umweltbundesamt \cite{Umweltbundesamt.7.10.2021} recommends a room temperature of 17°C during the absence of any person. Therefore, we consider scenario 2 with the desired $T_\text{inside}$ during the presence of occupants at 22°C and during absence at 17°C. In opposite to the basic scenario, we do not change the cost function \ref{eq:costfunctatsächlich} during presence and absence. We switch the $y_\text{track}$ between a $y_\text{track,presence}$ as 22°C and a $y_\text{track,absence}$ as 17°C. We achieve the implementation of scenario 2 using one more symbolic parameter in the optimisation, which is set by the occupancy schedule to $y_\text{track,presence}$ or $y_\text{track,absence}$. In addition, the temperature range needs to enlarge because the lower bound requires to be under the requested temperature. Thus, we choose 16.5°C as a lower bound. The set of constraints changes to:
\begin{equation}
    \label{ConstraintYScenario2}
    \mathbb{Y_k} = \{\mathbf{y_k}| 16.5 \text{°C} - \eta_k \prec T_\text{inside} \prec 26 \text{°C}+ \eta_k\} 
\end{equation}
The MPC algorithm has to handle significant fluctuations of the desired temperature. Therefore, we reduce the weighting $w_\text{3}$ to minimum needed depending on the $w_\text{1}$ and $w_\text{2}$ and adjust the soft constraint. In this way, we allow more deviations of the desired temperature to obtain a feasible solution of the MPC. \newline 

\textbf{(iii) Scenario 3:}\newline
Scenario 3 is the simplest optimisation problem. The desired $T_\text{inside}$ is 22°C over the whole simulation time. This case does also not consider the occupancy schedule. We can use the cost function \ref{eq:costfunctatsächlich} without changes during this time and all constraints discussed in \autoref{section:theconstraints}.\newline

To summarise the different desired behaviours of the scenarios the desired temperatures are pictured in the following Figure.
    \begin{figure}[H]
           \centering
        \def\svgwidth{1.1\textwidth}
        \input{figure/Solltemperaturverlauf.pdf_tex}
        \caption{Desired temperature trajectories of the scenarios}
         \label{fig:Solltemperaturverlauf}
    \end{figure}

\subsection{Results of the scenarios}
\label{subsec:Results of the scenarios}
To point out the influence of the weightings in the cost function, all results are presented for two representative weights, which enables to focus on the comfort requirements or grid services. The chosen weightings are $i_\text{1} = 0.1 \vee 0.9$ and $i_\text{2} = 0.9 \vee 0.1$. In the following the resulting inside temperature $T_\text{inside}$, the control signals  $\dot{Q}_\text{heating}$ and $\dot{Q}_\text{HP}$, and the electrical consumption of the heat pump $P_\text{HP}$ are shown over the simulation period, as well as the $\Delta \overline{y}$ and GS for every scenario.\newline

\autoref{fig:TemperaturverlaufScenarien} shows the curves of $T_\text{inside}$ of the three different scenarios over nine days, the simulation period. The chosen unit for temperatures is Kelvin. After converting  the tracking temperatures to Kelvin, we obtain $y_\text{track} = 295.15 K$, $y_\text{track,presence} = 295.15 K$, and $y_\text{track,absence} = 290.15 K$. We separate between the higher grid service $i_\text{2} = 0.9$ and lower comfort requirement $i_\text{1} = 0.1$ in the above diagram and the lower grid service $i_\text{1} = 0.1$ and higher comfort requirement $i_\text{1} = 0.9$ in the below diagram.\newline 
The inside temperature curve of scenario 2 runs below $T_\text{inside}$ of both other scenarios. Scenario 3 is closer to the $y_\text{track}$ than the basic scenario. 
    \begin{figure}[H]
           \centering
        \def\svgwidth{1\textwidth}
        \input{figure/TemperaturverlaufScenarien.pdf_tex}
        \caption{$T_\text{inside}$ for the three scenarios for $i_\text{1} = 0.1$ and $i_\text{2} = 0.9$ and $i_\text{1} = 0.9$ and $i_\text{2} = 0.1$}
         \label{fig:TemperaturverlaufScenarien}
    \end{figure}
 
In \autoref{fig:HeizverlaufGewichte}, $\dot{Q}_\text{heating}$ is depicted over the simulation period. Also, the weightings are distinguished in two diagrams. We see negative and positive values for $\dot{Q}_\text{heating}$ in all scenarios and further some constant values.
     \begin{figure}[H]
           \centering
        \def\svgwidth{1.05\textwidth}
        \input{figure/HeizverlaufGewichte.pdf_tex}
        \caption{$\dot{Q}_\text{heating}$ for the three scenarios for $i_\text{1} = 0.1$ and $i_\text{2} = 0.9$ and $i_\text{1} = 0.9$ and $i_\text{2} = 0.1$}
         \label{fig:HeizverlaufGewichte}
    \end{figure}
    
\autoref{fig:HPwaermestroeme} illustrates the heat flow $\dot{Q}_\text{HP}$ for all scenarios over the simulation time with both weightings. We notice that scenario 2 differs stronger from the other scenarios and the basic scenario has a constant period from day three to five. The values are in both diagrams positive and negative and we see higher absolute values in the second diagram with $i_\text{1} = 0.9$ and $i_\text{2} = 0.1$.

    \begin{figure}[H]
           \centering
        \def\svgwidth{1.01\textwidth}
        \input{figure/HPwaermestroeme.pdf_tex}
        \caption{$\dot{Q}_\text{HP}$ for the three scenarios for $i_\text{1} = 0.1$ and $i_\text{2} = 0.9$ and $i_\text{1} = 0.9$ and $i_\text{2} = 0.1$}
         \label{fig:HPwaermestroeme}
    \end{figure}
    
In the following \autoref{fig:HP_grid} the structure for the comparison between the scenarios differs from the Figures above. Here, we separate every scenario in its diagram and vary the weightings within the diagram. The first diagram shows the curve of the electricity consumption of the heat pump for the basic scenario with $i_\text{1} = 0.1 and i_\text{2} = 0.9$ and $i_\text{1} = 0.9 and i_\text{2} = 0.1$ during the simulation period. Further, the dynamic electricity price $Pr$ is illustrated at the same time. Both further diagrams depict the same for the other scenarios. $Pr$ is independent of the scenarios. Therefore, we see the same curves three times in every diagram. The curves of $P_\text{HP}$ with lower weighting on grid services run above those with higher weighting. Further, there are some yellow highlights at the x-axis, which are needed for a better discussion below. 
    \begin{figure}[H]
           \centering
        \def\svgwidth{0.9\textwidth}
        \input{figure/HP_grid2.pdf_tex}
        \caption{$P_\text{HP}$ and $Pr$ for the three scenarios for $i_\text{1} = 0.1$ and $i_\text{2} = 0.9$ and $i_\text{1} = 0.9$ and $i_\text{2} = 0.1$}
         \label{fig:HP_grid}
    \end{figure}
    
\autoref{tab:AverageComfortScenarien} presents the average comfort $\Delta \overline{y}$ of the scenarios calculated according to \autoref{eq:average comfort}. We also differentiate the weighting of comfort with $i_\text{1} =$ 0.1 or 0.9. Scenario 2 has the highest values of $\Delta \overline{y}$ and the basic scenario and scenario 3 changes the order with the weighting. Only scenario 2 has a high $\Delta \overline{y}$ with higher weighting of comfort.  
    \begin{table}[H]
        \centering
        \begin{tabular}{c||c|c|c}
          $i_\text{1}$  &  Basic scenario & Scenario 2 & Scenario 3\\
          \hline  \hline
             0.1 & 0.77 & 1.46 & 0.55\\
             0.9 & 0.13 & 2.02 & 0.17\\
        \end{tabular}
        \caption{$\Delta \overline{y}$ of the scenarios according to $i_\text{1}$}
        \label{tab:AverageComfortScenarien}
    \end{table}
    
\autoref{tab:GridserviceScenarien} shows the equivalent of the table above but grid services (GS) is of interest, also with two weightings  $i_\text{2} =$ 0.1 or 0.9. The values of GS are presented for every scenario calculated according to \autoref{eq:GridService123}. The lower values of GS depend on higher weighting of grid services. 
    \begin{table}[H]
        \centering
        \begin{tabular}{c||c|c|c}
          $i_\text{2}$  &  Basic scenario & Scenario 2 & Scenario 3\\
          \hline  \hline
             0.1 & 9.00 & 10.40 & 9.69 \\
             0.9 & 2.85 & 6.36 & 4.57\\
        \end{tabular}
        \caption{GS of the scenarios according to $i_\text{2}$}
        \label{tab:GridserviceScenarien}
    \end{table}
    
\autoref{tab:EnergyConsumptionScenarien} depicts the energy consumption in kWh as sum of the power of $P_\text{HP}$ over the simulation period. We separate the weighting consideration of every scenario. All scenarios have higher energy consumption with higher weighting on the comfort requirement in the cost function. The basic scenario has the lowest values. 
    \begin{table}[H]
        \centering
        \begin{tabular}{c||c|c|c}
          $i_\text{1} \ \& \  i_\text{2}$  &  Basic scenario & Scenario 2 & Scenario 3\\
          \hline  \hline
             0.1 \& 0.9 & 5.7 MWh & 15.6 MWh & 9.9 MWh\\
             0.9 \& 0.1 & 18.5 MWh & 23.7 MWh & 21.0 MWh\\
        \end{tabular}
        \caption{Energy consumption of the scenarios according to the weighting}
        \label{tab:EnergyConsumptionScenarien}
    \end{table}

\section{Discussion}
\label{sec:discussion}
The presented results from \autoref{subsec:Results of the scenarios} are explained and analysed in this section. In addition, the general realisation of this thesis is discussed.

\subsection{Comparison of the scenarios}
\label{subsec:Comparison fo the scenarios}
To compare the different scenarios, we use the results in the section above. We comment every Figure and Table from \autoref{subsec:Results of the scenarios} and analyse comfort, grid services and energy consumption.\newline

\textbf{Comfort}\newline
Comfort is an important requirement in the MPC to maintain a pleasant temperature in the building for the occupants. Therefore, a detailed interpretation of the scenarios follows focusing on comfort.\newline
We observe in the inside temperature curves from \autoref{fig:TemperaturverlaufScenarien} that the desired temperature $y_\text{track}$ of the scenarios cannot be reached. For a better analysis, we consider \autoref{fig:HeizverlaufGewichte} with the control signal $\dot{Q}_{heating}$. The heat flow $\dot{Q}_{heating}$ often remains constant at maximum or minimum, which means reaching the limits of the constraints. Although the absolute control signal is maximum, there are deviations from $y_\text{track}$. This behaviour indicates that the thermal model reacts not strong enough on the given control signals.\newline
Obviously, $T_\text{inside}$ of scenario 2 runs below the other scenarios. The reason for that follows from the changing desired temperature between 290.15 K and 295.15 K. Since in scenario 2, the comfort requirement in the cost function during the presence and the absence of occupants is weighted the same, the optimiser does not differentiate between these two cases and generates the best for both. It follows the inferior performance of the average comfort in \autoref{tab:AverageComfortScenarien} compared to the other scenarios. It is unusual that a higher weighting on comfort in the cost function results in a higher $\Delta \overline{y}$. By way of explanation, we first note that the absence phase of occupants is longer than the presence phase of occupants, which is predominantly due to the weekend work break. Focusing the comfort requirement in the cost function of scenario 2, the deviation of $T_\text{inside}$ from the desired temperature during absence $y_\text{track,absence}$ is lower due to the longer phase. This increases the difference between the actual temperature and the desired temperature during the presence phase, which cannot be compensated by the control signal. Consequently, we obtain a poorer average comfort with a higher weighting of the comfort requirement in scenario 2 (see \autoref{tab:AverageComfortScenarien}).\newline 
$T_\text{inside}$ of the basic scenario and scenario 3 run close together, with scenario 3 always closer to $y_\text{track} = 295.15 K$ (\autoref{fig:TemperaturverlaufScenarien}). A constant desired temperature over the whole day affects consequently positive on the comfort. We note also this for the $\Delta \overline{y}$ with the lower weighting $i_\text{1}$ in \autoref{tab:AverageComfortScenarien}. Also, it is visible in \autoref{fig:TemperaturverlaufScenarien} and \autoref{tab:AverageComfortScenarien} that a higher weighting of comfort effects the expected improvement of the comfort. We notice a slightly better $\Delta \overline{y}$ of the basic scenario of 0.04 K than scenario 3 with $i_\text{2}$. This results from the calculation of $\Delta \overline{y}$ wherein only the presence of occupants is considered. Due to the consideration of scenario 3 without an occupancy schedule, it is assumed that occupants are always in the building and that comfort must always be maintained. Thus, we sum more temperature deviations for obtaining $\Delta \overline{y}$ of scenario 3 and divide it by a higher number of $n_\text{occ}$. Nevertheless, this results in a slight decline of $\Delta \overline{y}$ compared to the basic scenario with the weighting $i_\text{1} = 0.9$. \newline

\textbf{Grid Services and energy consumption}\newline
To realise DSM with the heat pump in the reference building, we desire a reaction of the heat pump during the favourable times for the grid. Therefore, we analyse the grid services of the scenarios in the following.\newline
\autoref{fig:HeizverlaufGewichte} and \autoref{fig:HPwaermestroeme} depict the optimised control signals. We note that the optimiser avoids different signs of $\dot{Q}_\text{HP}$ and $\dot{Q}_\text{heating}$ so that the water reservoir and the building is heated or cooled, as it is constrained. Further, we observe a constant phase of $\dot{Q}_\text{HP}$ between the fourth and the sixth day of the basic scenario in \autoref{fig:HPwaermestroeme}. These days fall on the weekend, which means a longer phase without the comfort requirement in the cost function of the basic scenario. The phase exceeds the length of the predictive horizon, wherefore no energy has to be stored at favorable times according to the forecasting calculations. Therefore, the optimiser chooses $\dot{Q}_\text{HP} = 0$ to enable zero costs in the cost functions.\newline
The electrical consumption of the heat pump $P_\text{HP}$ is more interesting than $\dot{Q}_\text{HP}$ because $P_\text{HP}$ influences directly the grid, which is investigated. However, $P_\text{HP}$ is calculated from $\dot{Q}_\text{HP}$ (see \autoref{subsec:charcteristicDiagramHP}).\newline
\autoref{fig:HP_grid} and \autoref{tab:GridserviceScenarien} provide clarification about the grid services of the scenarios. In both representations, it is obvious that a higher weighting on grid services requirement in the cost function leads to lower values of $\dot{Q}_\text{HP}$ (see \autoref{fig:HPwaermestroeme}) and consequently of $P_\text{HP}$ (see \autoref{fig:HP_grid}). Since the higher weighting $i_\text{2}$ would lead to higher costs with higher values for $\dot{Q}_\text{HP}$. To determine which scenario is more suitable for grid services, we examine the amplitudes of $P_\text{HP}$ at cheap or expensive electricity prices $Pr$. The yellow highlights in \autoref{fig:HP_grid} mark obvious spots where the heat pump reacts as desired: at low prices $Pr$, which indicate a higher available load on the grid, the heat pump reacts with a larger control signal and at higher $Pr$, with a lower control signal. In general, all scenarios behave according to the specifications of the grid. Because the basic scenario does not consume power at the weekend, it does not react to the grid for a longer time and can therefore be classified with less grid service than the other two scenarios. It also requires the least costs for the entire period of the simulation, as revealed by \autoref{tab:GridserviceScenarien}, while scenario 2 requires most of the costs. Scenario 2 shows the high amplitudes which implies the strongest reaction on $Pr$. Therefore, it is the best scenario for grid services. Scenario 2 is only interesting because of the investigation of grid services. Focusing on the energy consumption, scenario 2 wastes energy (see the highest energy consumption in \autoref{tab:EnergyConsumptionScenarien}). The basic scenario, with the occupancy schedule, requires the least energy. \newline

\textbf{Answering the research question}\newline
We can summarise the comparison of the scenarios and analysis of the results. Even if the thermal model does not react as well as desired, the research question is answerable. The inclusion of an occupancy schedule in the MPC improves the energy consumption proceeded as the basic scenario. The opposite of the basic scenario is scenario 3. More grid services can be reached by a time-independent cost function as scenario 2 or 3, where both requirements grid services and comfort are always of interest. Scenario 2 reacts most on grid services and scenario 3 most on comfort. Nevertheless, considering the energy consumption, neither of the two scenarios can be recommended. The basic scenario enables the compromise on energy consumption and grid services. At the same time, a higher weighting of comfort in relation to grid services can be recommended. This will give us more influence on the grid while increasing comfort.  

\subsection{General discussion about the approach}
\label{subsec:General discussion about the approach}
In this subsection, we comment further issues affecting all scenarios to give an analysis over the used approach and clarify the limitations.\newline

\textbf{feasible problem}\newline
When implementing the MPC, we notice that simple changes in the code can lead to infeasible problems. The model, the constraints, and the weightings of the cost function have a great influence on finding a solution. Therefore, all parameters have to be chosen with diligence.\newline

\textbf{cooling and heating}\newline
The approach enables cooling and heating of the reference building. Thus, one model can be implemented for winter and summer. However, the model must represent sufficiently well winter and summer. The used model is verified for summer, since no data was available in winter during this thesis.
The optimiser wins more degrees of freedom due to allowing cooling and heating. This degree of freedom can be used to react more often at the favourable time for the grid. For example, if the temperature is above the desired temperature and we have a favourable time to consume power, it is preferable for grid services to cool the building.
However, cooling and heating correspond not to the habitual user behaviour of residents in buildings within a few days. Then, this approach consumes more energy than a approach where only heating or cooling is allowed.


%This chapter is supposed to discuss your results. Point out what your results mean.
%What are the limitations of your approach, managerial implications or future impact?
%
%Explain the broader picture but be critical with your methods.
\chapter{Conclusion and outlook}
\label{ch:Conclusionandoutlook}

\section{Conclusion}
\label{sec:conclusion}
Increasing fluctuations due to more renewable energy sources, such as photovoltaics or wind power, make balancing the grid more difficult. More energy storage options and demand side management can remedy the situation. In this thesis, an MPC of the HVAC system of a building uses the thermal storage of the building and consumes energy with the heat pump in a favourable time for the grid. Moreover, the influence of an occupancy schedule is investigated on grid services.\newline
At first, a thermal model is generated, which predicts the future behaviour of the building's average inside temperature in the MPC. The complete model is separated into the submodels water reservoir and building model. The water reservoir is modelled with the white-box approach, while the building model is created as a grey-box model. For the last mentioned submodel, we identify the parameters, such as thermal resistances and capacitances, with data from the reference building and verify the building model with a verification data set. The data is generated with experiments. Therefore, electrical heaters are positioned in the reference building and heat the building. The data acquisition occurs via temperature sensors and sensors for recording the power consumption, which are permanently installed in the building. The collected data is used for training and verification of the building model. \newline
In the MPC, the control signals from the heat pump and heating are optimised to regulate the indoor temperature over the predictive horizon with respect to the cost function. There, we consider constraints, which limit the control signal, pretend a temperature range or restrict states. To simulate the MPC, we use data from the reference building, which represent the weather forecasts. An occupancy schedule specifies the MPC the time when the comfort requirement is required. The dynamic price of electricity is used as an indicator for the favourable time to supply power from the grid. Furthermore, the length of the predictive horizon is determined as 24 h.\newline
At last, scenarios are investigated concerning comfort, grid service and energy consumption. The basic scenario includes the occupancy schedule and expects 22°C inside temperature during the presence of occupants. Scenario 2 respects the occupancy schedule and changes the desired temperature between 17°C and 22°C during the absence and presence of occupants. While scenario 3 always desires the inside temperature at 22°C. The basic scenario is best suited to combine comfort, energy consumption and grid services.\newline

\section{Outlook}
\label{sec:outlook}
%Repeat the problem and its relevance, as well as the contribution (plus quantitative results). 
%Look back at what you have written in the introduction.
%
%Provide an outlook for further research steps.
\chapter{Outlook}
\label{ch:outlook}
%During the implementation of the MPC, es hat sich herausgestellt, dass einfache veränderungen im Code schnell zu unlösbaren problemen führt. Das Modell, die constraints und die Gewichte der Kostenfunktion haben hier großen einfluss auf das finden einer lösung.


%% --------------------
%% |   Bibliography   |
%% --------------------

%% Add entry to the table of contents for the bibliography
\printbibliography[heading=bibintoc]
%% ----------------
%% |   Appendix   |
%% ----------------
\appendix
%% LaTeX2e class for student theses
%% appendix
%% 
%% Based on SDQ KIT Template by Erik Burger
%%
%% Karlsruhe Institute of Technology
%% Institute for Automation and Applied Informatics
%% AIDA Research Group
%%
%% Nicole Ludwig
%% nicole.ludwig@kit.edu
%%
%% Version 1.2, 2018-10-11


\iflanguage{english}
{\chapter{Appendix}}    % english style
{\chapter{Anhang}}      % german style
\label{chap:appendix}

\section{Model values}
\label{sec:appendix:Modelvalues}
\begin{table}[!]
    \centering
    \begin{tabular}{c|c|c}
         &  Initial values & Identified values\\
         \hline
        $C_\text{inside}$& 400425 $J/W$ & $J/W$\\
        $C_\text{envelope}$ & 24999045 $J/W$ & $J/W$\\
        $C_\text{interior}$& 22960754 $J/W$ & $J/W$\\
        $C_\text{floor}$ & 26118734 $J/W$ & $J/W$\\
        $R_\text{inside}$ & 191 $K/W$ & $K/W$\\
        $R_\text{window}$ & 34 $K/W$ & $K/W$\\
        $R_\text{envelope}$ & 287 $K/W$ & $K/W$\\
        $R_\text{interior}$& 77 $K/W$ & $K/W$\\
        $R_\text{floor}$ & 749 $K/W$ & $K/W$\\
        $R_\text{in}$ & 191 $K/W$ & $K/W$\\
        $f_\text{sol,inside}$ & 0,25 & $K/W$\\
        $f_\text{sol,envelope}$ &0,25 & $K/W$\\
    \end{tabular}
    \caption{Initial and identified values of the model parameters}
    \label{tab:my_label}
\end{table}
\section{Matrices of state-space formulation}
\label{sec:appendix:Matrizen}
    
    \begin{equation}
    B_1 = 
    \begin{pmatrix}
        1 & 0 \\
        0 & 0 \\
        0 & 0 \\
        0 & 0 \\
        -1 & 1 
    \end{pmatrix}
    \end{equation}
    \begin{equation}
	B_2
	\begin{pmatrix}
         f_{sun,inside} & 0 & \frac{1}{C_{inside}R_{window}} & 0\\
         0 & f_{sun,envelope} & \frac{1}{C_{envelope}R_{envelope}}&0\\
         0 & 0 & 0& 0\\
         0 & 0 & 0& 0\\
         0 & 0 & 0 & -1
    \end{pmatrix}
	\end{equation}
	\begin{equation}
	    C = 
	    \begin{pmatrix}
        1 & 0 & 0 & 0 & 0 \\
        \end{pmatrix}
	\end{equation}
	\begin{landscape}
	\begin{align} 
	A = \nonumber \\ 
	\begin{pmatrix}
    \frac{-1}{C_{inside}R_{inside}}-\frac{1}{C_{inside}R_{window}}-\frac{1}{C_{inside}R_{interior}}-\frac{1}{C_{inside}R_{floor}}   & \frac{1}{C_{inside}R_{inside}} & \frac{1}{C_{inside}R_{interior}} & \frac{1}{C_{inside}R_{floor}} & 0 \\
    \frac{1}{C_{envelope}R_{inside}}& \frac{-1}{C_{envelope}R_{envelope}}- \frac{1}{C_{envelope}R_{inside}} & 0 & 0 & 0 \\
    \frac{1}{C_{interior}R_{interior}} & 0 & -\frac{1}{C_{interior}R_{interior}} & 0 &0 \\
    \frac{1}{C_{floor}R_{floor}} & 0 & 0 & -\frac{1}{C_{floor}R_{floor}} &0 \\
    0 & 0 & 0 & 0 & 0
    \end{pmatrix} \nonumber\\ 
    \end{align}
    \end{landscape}



\setcounter{figure}{0}
		
\begin{figure} [ht]
  \centering
  \caption{A figure}
  \label{fig:anotherfigure}
\end{figure}


\dots

\end{document}