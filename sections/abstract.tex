\pdfbookmark[1]{Abstract}{Abstract}

\chapter{Abstract}

The increasing of renewable energy sources into the grid leads to more electricity fluctuations in the power generation. Demand side management (DSM) can be an instrument to compensate for the fluctuations. The use of DSM for heat pumps in residential buildings has potential. However, the challenge is to achieve user comfort via thermal storage in a water reservoir or the thermal capacity of the building itself. Model predictive control (MPC) is a suitable tool to realize DSM by integrating forecasts.\newline
The thesis presents a simulative investigation of an MPC of a heat pump for a real-world building from the Energy Lab 2.0 at the KIT. A thermal building model is created with measurements from the reference building applying the grey-box approach. To record the training data and the verification data, the building is heated with electrical heaters in two experiments.\newline
An MPC is generated concerning comfort and grid services requirements in the cost function. The comfort requirement depends on the occupancy schedule of the reference building. In the case of the presence of occupants, a desired comfortable temperature is dictated, while in case of absence only the grid service is of interest. This MPC configuration is called the basic scenario. Further scenarios are investigated to evaluate the influence of an occupancy schedule on comfort and grid services. A concluding discussion shows that the occupancy schedule impacts positively the comfort requirement with less energy consumption and low costs for grid services when using the basic scenario.