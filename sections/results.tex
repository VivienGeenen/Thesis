\chapter{Results and discussion}
\label{ch:results}
In this chapter, different scenarios are considered to answer the research question focusing on grid services. Several scenarios are supposed to explain how the occupancy schedule affects energy consumption, grid service and comfort: (i) a scenario with an occupancy schedule, in which the MPC has to comply with fewer restrictions during the absence of occupants; (ii) a scenario that also has a lower temperature specification during the absence; and (iii) a scenario without an occupancy schedule. In the last section, the results are evaluated, and an answer to the research question is found.

\section{Results of the scenarios}
\label{sec:ResultsScenarios}
In this section, the implementation of the different scenarios is explained and the results of the temperature and control signal curves are presented. Furthermore, energy consumption, comfort and grid services are evaluated.  

\subsection{Presentation of the scenarios}
\label{subsec:Presentation of the scenarios}

\textbf{(i) The basic scenario:}\newline
The first considered scenario is the basic scenario, which is explained in detail in \autoref{ch:mpc}. Summarised, we desire the $T_\text{inside}$ during the presence of occupants as 22°C. And we leave the optimiser free to find an optimal temperature in the specified temperature range during the absence realised by neglecting the comfort requirement in the cost function.\newline

\textbf{(ii) Scenario 2:}\newline
The Umweltbundesamt \cite{Umweltbundesamt.7.10.2021} recommends a room temperature of 17°C during the absence of any person. Therefore, we consider scenario 2 with the desired $T_\text{inside}$ during the presence of occupants at 22°C and during absence at 17°C. In opposite to the basic scenario, we do not change the cost function \ref{eq:costfunctatsächlich} during presence and absence. We switch the $y_\text{track}$ between a $y_\text{track,presence}$ as 22°C and a $y_\text{track,absence}$ as 17°C. We achieve the implementation of scenario 2 using one more symbolic parameter in the optimisation, which is set by the occupancy schedule to $y_\text{track,presence}$ or $y_\text{track,absence}$. In addition, the temperature range needs to enlarge because the lower bound requires to be under the requested temperature. Thus, we choose 16.5°C as a lower bound. The set of constraints changes to:
\begin{equation}
    \label{ConstraintYScenario2}
    \mathbb{Y_k} = \{\mathbf{y_k}| 16.5 \text{°C} - \eta_k \prec T_\text{inside} \prec 26 \text{°C}+ \eta_k\} 
\end{equation}
The MPC algorithm has to handle significant fluctuations of the desired temperature. Therefore, we reduce the weighting $w_\text{3}$ to minimum needed depending on the $w_\text{1}$ and $w_\text{2}$ and adjust the soft constraint. In this way, we allow more deviations of the desired temperature to obtain a feasible solution of the MPC. \newline 

\textbf{(iii) Scenario 3:}\newline
Scenario 3 is the simplest optimisation problem. The desired $T_\text{inside}$ is 22°C over the whole simulation time. This case does also not consider the occupancy schedule. We can use the cost function \ref{eq:costfunctatsächlich} without changes during this time and all constraints discussed in \autoref{section:theconstraints}.

    \begin{figure}[H]
           \centering
        \def\svgwidth{1\textwidth}
        \input{figure/Solltemperaturverlauf.pdf_tex}
        \caption{}
         \label{fig:Solltemperaturverlauf}
    \end{figure}

\subsection{Results of the scenarios}
\label{subsec:Results of the scenarios}
To point out the influence of the weightings in the cost function, all results are presented for two representative weights, which enables to focus on the comfort requirements or grid services. The chosen weightings are $i_\text{1} = 0.1 \vee 0.9$ and $i_\text{2} = 0.9 \vee 0.1$. In the following the resulting inside temperature $T_\text{inside}$, the control signals  $\dot{Q}_\text{heating}$ and $\dot{Q}_\text{HP}$, and the electrical consumption of the heat pump $P_\text{HP}$ are shown over the simulation period, as well as the $\Delta \overline{y}$ and GS for every scenario.\newline

\autoref{fig:TemperaturverlaufScenarien} shows the curves of $T_\text{inside}$ of the three different scenarios over nine days, the simulation period. The chosen unit for temperatures is Kelvin. After converting  the tracking temperatures to Kelvin, we obtain $y_\text{track} = 295.15 K$, $y_\text{track,presence} = 295.15 K$, and $y_\text{track,absence} = 290.15 K$. We separate between the higher grid service $i_\text{2} = 0.9$ and lower comfort requirement $i_\text{1} = 0.1$ in the above diagram and the lower grid service $i_\text{1} = 0.1$ and higher comfort requirement $i_\text{1} = 0.9$ in the below diagram.\newline 
The inside temperature curve of scenario 2 runs below $T_\text{inside}$ of both other scenarios. Scenario 3 is closer to the $y_\text{track}$ than the basic scenario. 
    \begin{figure}[H]
           \centering
        \def\svgwidth{1\textwidth}
        \input{figure/TemperaturverlaufScenarien.pdf_tex}
        \caption{$T_\text{inside}$ for the three scenarios for $i_\text{1} = 0.1$ and $i_\text{2} = 0.9$ and $i_\text{1} = 0.9$ and $i_\text{2} = 0.1$}
         \label{fig:TemperaturverlaufScenarien}
    \end{figure}
 
In \autoref{fig:HeizverlaufGewichte}, $\dot{Q}_\text{heating}$ is depicted over the simulation period. Also, the weightings are distinguished in two diagrams. We see negative and positive values for $\dot{Q}_\text{heating}$ in all scenarios and further some constant values.
     \begin{figure}[H]
           \centering
        \def\svgwidth{1.05\textwidth}
        \input{figure/HeizverlaufGewichte.pdf_tex}
        \caption{$\dot{Q}_\text{heating}$ for the three scenarios for $i_\text{1} = 0.1$ and $i_\text{2} = 0.9$ and $i_\text{1} = 0.9$ and $i_\text{2} = 0.1$}
         \label{fig:HeizverlaufGewichte}
    \end{figure}
    
\autoref{fig:HPwaermestroeme} illustrates the heat flow $\dot{Q}_\text{HP}$ for all scenarios over the simulation time with both weightings. We notice that scenario 2 differs stronger from the other scenarios and the basic scenario has a constant period from day three to five. The values are in both diagrams positive and negative and we see higher absolute values in the second diagram with $i_\text{1} = 0.9$ and $i_\text{2} = 0.1$.

    \begin{figure}[H]
           \centering
        \def\svgwidth{1.01\textwidth}
        \input{figure/HPwaermestroeme.pdf_tex}
        \caption{$\dot{Q}_\text{HP}$ for the three scenarios for $i_\text{1} = 0.1$ and $i_\text{2} = 0.9$ and $i_\text{1} = 0.9$ and $i_\text{2} = 0.1$}
         \label{fig:HPwaermestroeme}
    \end{figure}
    
In the following \autoref{fig:HP_grid} the structure for the comparison between the scenarios differs from the Figures above. Here, we separate every scenario in its diagram and vary the weightings within the diagram. The first diagram shows the curve of the electricity consumption of the heat pump for the basic scenario with $i_\text{1} = 0.1 and i_\text{2} = 0.9$ and $i_\text{1} = 0.9 and i_\text{2} = 0.1$ during the simulation period. Further, the dynamic electricity price $Pr$ is illustrated at the same time. Both further diagrams depict the same for the other scenarios. $Pr$ is independent of the scenarios. Therefore, we see the same curves three times in every diagram. The curves of $P_\text{HP}$ with lower weighting on grid services run above those with higher weighting. Further, there are some yellow highlights at the x-axis, which are needed for a better discussion below. 
    \begin{figure}[H]
           \centering
        \def\svgwidth{0.9\textwidth}
        \input{figure/HP_grid2.pdf_tex}
        \caption{$P_\text{HP}$ and $Pr$ for the three scenarios for $i_\text{1} = 0.1$ and $i_\text{2} = 0.9$ and $i_\text{1} = 0.9$ and $i_\text{2} = 0.1$}
         \label{fig:HP_grid}
    \end{figure}
    
\autoref{tab:AverageComfortScenarien} presents the average comfort $\Delta \overline{y}$ of the scenarios calculated according to \autoref{eq:average comfort}. We also differentiate the weighting of comfort with $i_\text{1} =$ 0.1 or 0.9. Scenario 2 has the highest values of $\Delta \overline{y}$ and the basic scenario and scenario 3 changes the order with the weighting. Only scenario 2 has a high $\Delta \overline{y}$ with higher weighting of comfort.  
    \begin{table}[H]
        \centering
        \begin{tabular}{c||c|c|c}
          $i_\text{1}$  &  Basic scenario & Scenario 2 & Scenario 3\\
          \hline  \hline
             0.1 & 0.77 & 1.46 & 0.55\\
             0.9 & 0.13 & 2.02 & 0.17\\
        \end{tabular}
        \caption{$\Delta \overline{y}$ of the scenarios according to $i_\text{1}$}
        \label{tab:AverageComfortScenarien}
    \end{table}
    
\autoref{tab:GridserviceScenarien} shows the equivalent of the table above but grid services (GS) is of interest, also with two weightings  $i_\text{2} =$ 0.1 or 0.9. The values of GS are presented for every scenario calculated according to \autoref{eq:GridService123}. The lower values of GS depend on higher weighting of grid services. 
    \begin{table}[H]
        \centering
        \begin{tabular}{c||c|c|c}
          $i_\text{2}$  &  Basic scenario & Scenario 2 & Scenario 3\\
          \hline  \hline
             0.1 & 9.00 & 10.40 & 9.69 \\
             0.9 & 2.85 & 6.36 & 4.57\\
        \end{tabular}
        \caption{GS of the scenarios according to $i_\text{2}$}
        \label{tab:GridserviceScenarien}
    \end{table}
    
\autoref{tab:EnergyConsumptionScenarien} depicts the energy consumption in kWh as sum of the power of $P_\text{HP}$ over the simulation period. We separate the weighting consideration of every scenario. All scenarios have higher energy consumption with higher weighting on the comfort requirement in the cost function. The basic scenario has the lowest values. 
    \begin{table}[H]
        \centering
        \begin{tabular}{c||c|c|c}
          $i_\text{1} \ \& \  i_\text{2}$  &  Basic scenario & Scenario 2 & Scenario 3\\
          \hline  \hline
             0.1 \& 0.9 & 5.7 MWh & 15.6 MWh & 9.9 MWh\\
             0.9 \& 0.1 & 18.5 MWh & 23.7 MWh & 21.0 MWh\\
        \end{tabular}
        \caption{Energy consumption of the scenarios according to the weighting}
        \label{tab:EnergyConsumptionScenarien}
    \end{table}

\section{Discussion}
\label{sec:discussion}
The presented results from \autoref{subsec:Results of the scenarios} are explained and analysed in this section. In addition, the general realisation of this thesis is discussed.

\subsection{Comparison of the scenarios}
\label{subsec:Comparison fo the scenarios}
To compare the different scenarios, we use the results in the section above. We comment every Figure and Table from \autoref{subsec:Results of the scenarios} and analyse comfort, grid services and energy consumption.\newline

\textbf{Comfort}\newline
Comfort is an important requirement in the MPC to maintain a pleasant temperature in the building for the occupants. Therefore, a detailed interpretation of the scenarios follows focusing on comfort.\newline
We observe in the inside temperature curves from \autoref{fig:TemperaturverlaufScenarien} that the desired temperature $y_\text{track}$ of the scenarios cannot be reached. For a better analysis, we consider \autoref{fig:HeizverlaufGewichte} with the control signal $\dot{Q}_{heating}$. The heat flow $\dot{Q}_{heating}$ often remains constant at maximum or minimum, which means reaching the limits of the constraints. Although the absolute control signal is maximum, there are deviations from $y_\text{track}$. This behaviour indicates that the thermal model reacts not strong enough on the given control signals.\newline
Obviously, $T_\text{inside}$ of scenario 2 runs below the other scenarios. The reason for that follows from the changing desired temperature between 290.15 K and 295.15 K. Since in scenario 2, the comfort requirement in the cost function during the presence and the absence of occupants is weighted the same, the optimiser does not differentiate between these two cases and generates the best for both. It follows the inferior performance of the average comfort in \autoref{tab:AverageComfortScenarien} compared to the other scenarios. It is unusual that a higher weighting on comfort in the cost function results in a higher $\Delta \overline{y}$. By way of explanation, we first note that the absence phase of occupants is longer than the presence phase of occupants, which is predominantly due to the weekend work break. Focusing the comfort requirement in the cost function of scenario 2, the deviation of $T_\text{inside}$ from the desired temperature during absence $y_\text{track,absence}$ is lower due to the longer phase. This increases the difference between the actual temperature and the desired temperature during the presence phase, which cannot be compensated by the control signal. Consequently, we obtain a poorer average comfort with a higher weighting of the comfort requirement in scenario 2 (see \autoref{tab:AverageComfortScenarien}).\newline 
$T_\text{inside}$ of the basic scenario and scenario 3 run close together, with scenario 3 always closer to $y_\text{track} = 295.15 K$ (\autoref{fig:TemperaturverlaufScenarien}). A constant desired temperature over the whole day affects consequently positive on the comfort. We note also this for the $\Delta \overline{y}$ with the lower weighting $i_\text{1}$ in \autoref{tab:AverageComfortScenarien}. Also, it is visible in \autoref{fig:TemperaturverlaufScenarien} and \autoref{tab:AverageComfortScenarien} that a higher weighting of comfort effects the expected improvement of the comfort. We notice a slightly better $\Delta \overline{y}$ of the basic scenario of 0.04 K than scenario 3 with $i_\text{2}$. This results from the calculation of $\Delta \overline{y}$ wherein only the presence of occupants is considered. Due to the consideration of scenario 3 without an occupancy schedule, it is assumed that occupants are always in the building and that comfort must always be maintained. Thus, we sum more temperature deviations for obtaining $\Delta \overline{y}$ of scenario 3 and divide it by a higher number of $n_\text{occ}$. Nevertheless, this results in a slight decline of $\Delta \overline{y}$ compared to the basic scenario with the weighting $i_\text{1} = 0.9$. \newline

\textbf{Grid Services and energy consumption}\newline
To realise DSM with the heat pump in the reference building, we desire a reaction of the heat pump during the favourable times for the grid. Therefore, we analyse the grid services of the scenarios in the following.\newline
\autoref{fig:HeizverlaufGewichte} and \autoref{fig:HPwaermestroeme} depict the optimised control signals. We note that the optimiser avoids different signs of $\dot{Q}_\text{HP}$ and $\dot{Q}_\text{heating}$ so that the water reservoir and the building is heated or cooled, as it is constrained. Further, we observe a constant phase of $\dot{Q}_\text{HP}$ between the fourth and the sixth day of the basic scenario in \autoref{fig:HPwaermestroeme}. These days fall on the weekend, which means a longer phase without the comfort requirement in the cost function of the basic scenario. The phase exceeds the length of the predictive horizon, wherefore no energy has to be stored at favorable times according to the forecasting calculations. Therefore, the optimiser chooses $\dot{Q}_\text{HP} = 0$ to enable zero costs in the cost functions.\newline
The electrical consumption of the heat pump $P_\text{HP}$ is more interesting than $\dot{Q}_\text{HP}$ because $P_\text{HP}$ influences directly the grid, which is investigated. However, $P_\text{HP}$ is calculated from $\dot{Q}_\text{HP}$ (see \autoref{subsec:charcteristicDiagramHP}).\newline
\autoref{fig:HP_grid} and \autoref{tab:GridserviceScenarien} provide clarification about the grid services of the scenarios. In both representations, it is obvious that a higher weighting on grid services requirement in the cost function leads to lower values of $\dot{Q}_\text{HP}$ (see \autoref{fig:HPwaermestroeme}) and consequently of $P_\text{HP}$ (see \autoref{fig:HP_grid}). Since the higher weighting $i_\text{2}$ would lead to higher costs with higher values for $\dot{Q}_\text{HP}$. To determine which scenario is more suitable for grid services, we examine the amplitudes of $P_\text{HP}$ at cheap or expensive electricity prices $Pr$. The yellow highlights in \autoref{fig:HP_grid} mark obvious spots where the heat pump reacts as desired: at low prices $Pr$, which indicate a higher available load on the grid, the heat pump reacts with a larger control signal and at higher $Pr$, with a lower control signal. In general, all scenarios behave according to the specifications of the grid. Because the basic scenario does not consume power at the weekend, it does not react to the grid for a longer time and can therefore be classified with less grid service than the other two scenarios. It also requires the least costs for the entire period of the simulation, as revealed by \autoref{tab:GridserviceScenarien}, while scenario 2 requires most of the costs. Scenario 2 shows the high amplitudes which implies the strongest reaction on $Pr$. Therefore, it is the best scenario for grid services. Scenario 2 is only interesting because of the investigation of grid services. Focusing on the energy consumption, scenario 2 wastes energy (see the highest energy consumption in \autoref{tab:EnergyConsumptionScenarien}). The basic scenario, with the occupancy schedule, requires the least energy. \newline

\textbf{Answering the research question}\newline
We can summarise the comparison of the scenarios and analysis of the results. Even if the thermal model does not react as well as desired, the research question is answerable. The inclusion of an occupancy schedule in the MPC improves the energy consumption proceeded as the basic scenario. The opposite of the basic scenario is scenario 3. More grid services can be reached by a time-independent cost function as scenario 2 or 3, where both requirements grid services and comfort are always of interest. Scenario 2 reacts most on grid services and scenario 3 most on comfort. Nevertheless, considering the energy consumption, neither of the two scenarios can be recommended. The basic scenario enables the compromise on energy consumption and grid services. At the same time, a higher weighting of comfort in relation to grid services can be recommended. This will give us more influence on the grid while increasing comfort.  

\subsection{General discussion about the approach}
\label{subsec:General discussion about the approach}
In this subsection, we comment further issues affecting all scenarios to give an analysis over the used approach and clarify the limitations.\newline

\textbf{feasible problem}\newline
When implementing the MPC, we notice that simple changes in the code can lead to infeasible problems. The model, the constraints, and the weightings of the cost function have a great influence on finding a solution. Therefore, all parameters have to be chosen with diligence.\newline

\textbf{cooling and heating}\newline
The approach enables cooling and heating of the reference building. Thus, one model can be implemented for winter and summer. However, the model must represent sufficiently well winter and summer. The used model is verified for summer, since no data was available in winter during this thesis.
The optimiser wins more degrees of freedom due to allowing cooling and heating. This degree of freedom can be used to react more often at the favourable time for the grid. For example, if the temperature is above the desired temperature and we have a favourable time to consume power, it is preferable for grid services to cool the building.
However, cooling and heating correspond not to the habitual user behaviour of residents in buildings within a few days. Then, this approach consumes more energy than a approach where only heating or cooling is allowed.


%This chapter is supposed to discuss your results. Point out what your results mean.
%What are the limitations of your approach, managerial implications or future impact?
%
%Explain the broader picture but be critical with your methods.