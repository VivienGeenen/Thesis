\chapter{Results}
\label{ch:results}
In this chapter, different scenarios are considered to answer the research question focusing on grid services. Several scenarios are supposed to explain how the occupancy schedule affects energy consumption, grid service and comfort: (i) a scenario with an occupancy schedule, in which the MPC has to comply with fewer restrictions during the absence of occupants; (ii) a scenario that also has a lower temperature specification during the absence; and (iii) a scenario without an occupancy schedule. In the last section, the results are evaluated, and an answer to the research question is found.

\section{Results of the scenarios}
\label{sec:ResultsScenarios}
In this section, the implementation of the different scenarios is explained and the results of the temperature and control signal curves are presented. Furthermore, energy consumption, comfort and grid services are evaluated.  

\subsection{Presentation of the scenarios}
\label{subsec:Presentation of the scenarios}

\textbf{(i) The basic scenario:}\newline
The first considered scenario is the basic scenario, which is explained in detail in \autoref{ch:mpc}. Summarised, we desire the $T_\text{inside}$ during the presence of occupants as 22°C. And we leave the optimiser free to find an optimal temperature in the specified temperature range during the absence realised by neglecting the comfort requirement in the cost function.\newline

\textbf{(ii) Scenario 2:}\newline
The Umweltbundesamt \cite{Umweltbundesamt.7.10.2021} recommends during the absence of persons a room temperature of 17°C. Therefore, we consider scenario 2 with the desired $T_\text{inside}$ during the presence of occupants with 22°C and during absence with 17°C. In the opposite to the basic scenario, we do not change the cost function \ref{eq:costfunctatsächlich} during presence and absence. We switch the $y_\text{track}$ between a $y_\text{track,presence}$ as 22°C and a $y_\text{track,absence}$ as 17°C. We achieve the implementation of scenario 2 using one more symbolic parameter in the optimisation, which is set by the occupancy schedule to $y_\text{track,presence}$ or $y_\text{track,absence}$. In addition, the temperature range needs to enlarge because the lower bound requires to be under the requested temperature. Thus, we choose 16.5°C as a lower bound. The set of constraints changes to:
\begin{equation}
    \label{ConstraintYScenario2}
    \mathbb{Y_k} = \{\mathbf{y_k}| 16.5 \text{°C} - \eta_k \prec T_\text{inside} \prec 26 \text{°C}+ \eta_k\} 
\end{equation}
The MPC algorithm has to handle significant fluctuations of the desired temperature. Therefore, we reduce the weighting $w_\text{3}$ to minimum needed depending on the $w_\text{1}$ and $w_\text{2}$ and adjust the soft constraint. In this way, we allow more deviations of the desired temperature to obtain a feasible solution of the MPC. \newline 

\textbf{(iii) Scenario 3:}\newline
Scenario 3 is the simplest optimisation problem. The desired $T_\text{inside}$ is the 22°C over the complete simulation time. This case does also not consider the occupancy schedule. We can use the cost function \ref{eq:costfunctatsächlich} without changes during the time and all constraints discussed in \autoref{section:theconstraints}.

\subsection{Results of the scenarios}
\label{subsec:Results of the scenarios}
To point out the influence of the weightings in the cost function, all results are presented for two representative weights, which enables to focus on the requirements comfort or grid services. The chosen weightings are $i_\text{1} = 0.1 \vee 0.9$ and $i_\text{2} = 0.9 \vee 0.1$. In the following the resulting inside temperature $T_\text{inside}$, the heat flow $\dot{Q}_\text{heating}$, and the electrical consumption of the heat pump $P_\text{HP}$ are shown over the simulation period, as well as the AC and GS for every scenario.\newline

\autoref{fig:TemperaturverlaufScenarien} shows the curves of $T_\text{inside}$ of the three different scenarios over nine days, the simulation period. The chosen unit for temperatures is Kelvin. According to a conversion of the tracking temperatures in Kelvin, we obtain $y_\text{track} = 295.15 K$, $y_\text{track,presence} = 295.15 K$, and $y_\text{track,absence} = 290.15 K$. We separate between the higher grid service $i_\text{2} = 0.9$ and lower comfort requirement $i_\text{1} = 0.1$ in the above diagram and the lower grid service $i_\text{1} = 0.1$ and higher comfort requirement $i_\text{1} = 0.9$ in the below diagram.\newline 
The inside temperature curve of scenario 2 runs below $T_\text{inside}$ of both other scenarios. Scenario 3 is closer to the $y_\text{track}$ than the basic scenario. 
    \begin{figure}[H]
           \centering
        \def\svgwidth{1\textwidth}
        \input{figure/TemperaturverlaufScenarien.pdf_tex}
        \caption{$T_\text{inside}$ for the three scenarios for $i_\text{1} = 0.1 and i_\text{2} = 0.9$ and $i_\text{1} = 0.9 and i_\text{2} = 0.1$}
         \label{fig:TemperaturverlaufScenarien}
    \end{figure}
 
In \autoref{fig:HeizverlaufGewichte}, $\dot{Q}_\text{heating}$ is depicted over the simulation period. Also, the weightings are distinguished in two diagrams. We see negative and positive values for $\dot{Q}_\text{heating}$ in all scenarios and further some constant values.
     \begin{figure}[H]
           \centering
        \def\svgwidth{1.05\textwidth}
        \input{figure/HeizverlaufGewichte.pdf_tex}
        \caption{$\dot{Q}_\text{heating}$ for the three scenarios for $i_\text{1} = 0.1 and i_\text{2} = 0.9$ and $i_\text{1} = 0.9 and i_\text{2} = 0.1$}
         \label{fig:HeizverlaufGewichte}
    \end{figure}
    
    \begin{figure}[H]
           \centering
        \def\svgwidth{1\textwidth}
        \input{figure/HPwaermestroeme.pdf_tex}
        \caption{}
         \label{fig:HPwaermestroeme}
    \end{figure}
    
In the following \autoref{fig:HP_grid} the structure for the comparison between the scenarios differs from the Figures above. Here, we separate every scenario in its diagram and vary the weightings within the diagram. The first diagram shows the curve of the electricity consumption of the heat pump for the basic scenario with $i_\text{1} = 0.1 and i_\text{2} = 0.9$ and $i_\text{1} = 0.9 and i_\text{2} = 0.1$ during the simulation period. Further, the dynamic electricity price $Pr$ is illustrated at the same time. Both further diagrams depict the same for the other scenarios. $Pr$ is independent of the scenarios. Therefore, we see three times the same curves in every diagram. The curves of $P_\text{HP}$ with the lower weighting on grid services run above those with higher weighting on grid services. Further, some yellow highlights are at the x-axis, which is needed for a better discussion below. 
    \begin{figure}[H]
           \centering
        \def\svgwidth{0.9\textwidth}
        \input{figure/HP_grid2.pdf_tex}
        \caption{}
         \label{fig:HP_grid}
    \end{figure}
\autoref{tab:AverageComfortScenarien} presents the average comfort (AC) of the scenarios calculated according to \autoref{eq:average comfort}. We also differentiate the weighting of comfort with $i_\text{1} =$ 0.1 or 0.9. 
    \begin{table}[H]
        \centering
        \begin{tabular}{c||c|c|c}
          $i_\text{1}$  &  Basic scenario & Scenario 2 & Scenario 3\\
          \hline  \hline
             0.1 & 0.77 & 1.46 & 0.55\\
             0.9 & 0.13 & 2.02 & 0.17\\
        \end{tabular}
        \caption{Caption}
        \label{tab:AverageComfortScenarien}
    \end{table}
\autoref{tab:GridserviceScenarien} shows the equivalent of the table above but grid services (GS) is of interest, also with two weightings  $i_\text{2} =$ 0.1 or 0.9. The values of GS are presented for every scenario calculated according to \autoref{eq:GridService123}.
    \begin{table}[H]
        \centering
        \begin{tabular}{c||c|c|c}
          $i_\text{2}$  &  Basic scenario & Scenario 2 & Scenario 3\\
          \hline  \hline
             0.1 & 9.00 & 6.36 & 4.57\\
             0.9 & 2.85 & 9.00 & 10.40\\
        \end{tabular}
        \caption{Caption}
        \label{tab:GridserviceScenarien}
    \end{table}
%Verlauf temperatur und Stellgrößen, Angabe zu Energieverbrauch, Komfort und Grid services + nach verschiedenen Gewichten
\section{Discussion}
\label{sec:discussion}

\subsection{Comparison of the scenarios}
\label{subsec:Comparison fo the scenarios}

 % Nur wegen Netzdienlichkeit Scenario 2 überhaupt interessant. 
\subsection{General discussion about the approach}
\label{subsec:General discussion about the approach}

%This chapter is supposed to discuss your results. Point out what your results mean.
%What are the limitations of your approach, managerial implications or future impact?
%
%Explain the broader picture but be critical with your methods.