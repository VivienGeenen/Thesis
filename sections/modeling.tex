\chapter{Modelling}
\label{ch:modelling}
After explaining thermal basics and the electrical analogy, the foundations are used in this chapter. How I obtain a thermal model and the resulting thermal model, is introduced. Later the model is required for the MPC to predict the thermal reactions of the building. 

\section{The building}
\label{section:building}
Because of Da sich diese ganze arbeit auf ein reales Gebäude bezieht, wird im folgenden das Gebäude näher beschrieben.
\newline
The building stands at the "Campus Nord" from the "Karlsruher Institut für Technologie (KIT)" and is part of the "Energy Lab 2.0", "a research infrastructure for renewable energy"\cite{KIT.2021}. It is equipped with a kitchen, a bath room, five rooms and a technical room over with. In \autoref{fig:Bauplan}, you can see a part of the construction plan of the building

\begin{figure}
    \centering
    \def\svgwidth{320pt}
    \input{figure/Bauplan_WärmepumpenhausAuschnitt.pdf_tex}
    \caption{construction plan of the building \cite{Bauplan}}
    \label{fig:Bauplan}
\end{figure}
\section{The thermal model}
\label{thermalmodel}

\section{Data collection}

\section{Validation of the thermal Model}
\label{validationthermalmodel}



