\chapter{Modelling}
\label{ch:modelling}
After explaining thermal basics and the electrical analogy, the foundations are used in this chapter. The creation of the thermal model of the reference building and the resulting thermal model are presented. Later, the model is needed for the MPC to predict the thermal reactions of the building.

\section{The reference building}
\label{section:building}
    Since the thermal model and the MPC are based on a real building, details about the building are described below.
    \newline
    The building is located on the "Campus Nord" of the "Karlsruher Institut für Technologie (KIT)" and is part of the "Energy Lab 2.0", "a research infrastructure for renewable energy"\cite{KIT.2021}. It is equipped with a kitchen, a bathroom, five rooms and a technical room. For a better orientation, \autoref{fig:Bauplan} shows a part of the construction plan of the building. The building is designed as a single-family house, but for practical reasons, it is used as an office. The living area is around 100 $m^2$. The building offers two options to heat or cool with a ground-source heat pump or an air heat pump. The focus is on the air heat pump because the most commonly used heat pumps in Germany are air heat pumps \cite{bwp.2021}. In addition to the heat pump, there is a water reservoir for saving energy. The volume is 1000 litres \cite{Oskar}. However, the heating system is not completely installed yet. So heating or cooling is not possible actual.
    \newline
    One of the main features of the building is the number of sensors. The air temperature is measured in every room, as well as the temperature in the middle of the exterior wall, the screed temperature, the floor plate temperature, and the temperature of the inner wall between room three and room four (see \autoref{fig:Bauplan}). Furthermore, the consumption of the actual electrical power is also detected. Only the mentioned sensors are needed in this case, but there are many more sensors.
    %Himmelsrichtung noch beschreiben
    \begin{figure}
        \centering
        \def\svgwidth{320pt}
        \input{figure/Bauplan_WärmepumpenhausAuschnitt.pdf_tex}
        \caption{construction plan of the building \cite{Bauplan}}
        \label{fig:Bauplan}
    \end{figure}
    
\section{The thermal model}
\label{thermalmodel}
    %requirements
    The focus of this work is on the MPC part, so a simple thermal model is required. Nevertheless, no necessary information must be missing. Therefore, the thermal storage possibilities, the temperature inside of the building, and the heating system's influence have to be represented in the model. The storage allows to heat during the grid has too much power and to save the energy in the building during grid requires power. Hence, this enables the MPC's objective of grid services. The output of the model needs to be the temperature inside since the MPC aims to be in a pleasant temperature range to ensure customer comfort. Last, the influence of the heating system must be recognisable in the model, as it is the input of the plant.
    \newline
    The thermal model records the thermal conditions of the reference building. Therefore, the inner energy of the water reservoir and the air temperature inside the building are modelled. The water reservoir and the building behaviour are modelled according to different modelling strategies. The following chapters describe the partial models water reservoir and building model, the kind of modelling, and the conclusion of the partial models.
    
    
    \subsection{The modelling strategies}
    \label{ModellingStrategies}
    Creating a model can be made with three types of models, the so-called white-box models, grey-box models or black-box models. White-box models describe the real system only physically. Black-box models, on the other hand, have no physical description. They are created with data. And grey-box models are in between these two options \cite{Statusseminar.ForschungfurEnergieoptimiertesBauen.2009}. All possibilities are used in the thermal modelling of buildings \cite{Kramer.2012}.
    \newline
    The chosen approach for the MPC is the \textbf{grey-box model} for two reasons: First, this approach combines the advantages of white-box models and black-box models \cite{EstradaFlores.2006}. Second, there is the possibility to generate the required data from the reference building with the available measurement equipment at the KIT. According to Coakley et al., advantages and disadvantages are among other things\cite{Coakley.2014}:
    \begin{table}[h!]
    \label{Advantages and disadvantages of grey-box modelling}
        \centering
        \begin{tabular}{p{7.3cm} | p{7.3cm}}
        \hline
          Advantages  &  Disadvantages\\
        \hline
        \begin{itemize}
            \item faster development by a combination of physical and statistical model
        \end{itemize}
      & \begin{itemize}
            \item requires knowledge in physical and statistical modelling 
        \end{itemize}\\
     \begin{itemize}
            \item accuracy of the results for the specific use case, provided by qualitative training data
        \end{itemize} & \begin{itemize}
            \item changes at the building lead to a re-training
        \end{itemize}\\
        \end{tabular}
        \caption {Advantages and disadvantages of grey-box modelling}
    \end{table}
    \newline
    However, the water reservoir and the heating system are not in use yet. So, no data are available for training a grey-box model. That's the reason why a part of the model needs to create as \textbf{white-box model}.
    \newline
    But also, white-box models have some pros and cons (see the following Table \cite{EstradaFlores.2006}).
    \begin{table}[]
    \label{tab:wihte-boxpro}
        \centering
        \begin{tabular}{p{7.3cm} | p{7.3cm}}
        \hline
          Advantages  &  Disadvantages\\
        \hline
        \begin{itemize}
            \item relies on physics
        \end{itemize}
      & \begin{itemize}
            \item needs assumptions to simplify 
        \end{itemize}\\
     \begin{itemize}
            \item applicable for every situation with the same assumptions and requirements 
        \end{itemize} & \begin{itemize}
            \item often complex mathematical problems
        \end{itemize}\\
        \end{tabular}
        \caption {Advantages and disadvantages of white-box modelling}
    \end{table}


    \subsection{The water reservoir model}
    \label{waterModel}
    The water reservoire
    \begin{equation}
        \label{waterReservoir}
        \frac{d U_\text{WR}}{d t}= -\dot{Q}_\text{heating} + \dot{Q}_\text{HP} - \dot{Q}_\text{loss} - \dot{Q}_\text{SW}
    \end{equation}
    \begin{figure}
        \centering
        \def\svgwidth{150pt}
        \input{figure/Wasserspeicher.pdf_tex}
        \caption{Figure of the water reservoir with the heat flows}
        \label{fig:Bauplan}
    \end{figure}
    
    
    
    
    
    
 
    
   

    \subsection{The building model}
    \label{building model}

    \begin{figure}
            \centering
            \def\svgwidth{320pt}
            \input{figure/meinModel2.pdf_tex}
            \caption{structure of the thermal model in RC- analogy}
            \label{fig:structureThermalModel}
        \end{figure}
    
    \subsubsection{Data collection}
    \label{datacollection}

    \subsubsection{Validation of the thermal model}
    \label{validationthermalmodel}

    \subsection{The hole model/Zustandsraumdarstellung}
    \label{holeModel}

 \begin{align}
       \label{eq:meinModel2} 
       C_\text{inside}*\frac{d T_\text{inside}}{d t} &=& \dot{Q}_\text{heating} + \dot{Q}_\text{sun,inside} - \frac{T_\text{inside}-T_\text{envelope}}{R_\text{inside}} - \frac{T_\text{inside}-T_\text{outside}}{R_\text{window}} \\
       & &-\frac{T_\text{inside}-T_\text{interior}}{R_\text{interior}}-\frac{T_\text{inside}-T_\text{floor}}{R_\text{floor}} \nonumber\\
       C_\text{envelope}*\frac{d T_\text{envelope}}{d t} &=& \dot{Q}_\text{sun,envelope} - \frac{T_\text{envelope}-T_\text{outside}}{R_\text{envelope}} + \frac{T_\text{inside}-T_\text{envelope}}{R_\text{inside}} \nonumber \\
       C_\text{interior}*\frac{d T_\text{interior}}{d t} &=& \frac{T_\text{inside}-T_\text{interior}}{R_\text{interior}} \nonumber\\
       C_\text{floor}*\frac{d T_\text{floor}}{d t} &=& \frac{T_\text{inside}-T_\text{floor}}{R_\text{floor}} \nonumber\\
       \frac{d U_\text{WR}}{d t}&=& -\dot{Q}_\text{heating} + \dot{Q}_\text{HP} - \dot{Q}_\text{loss} - \dot{Q}_\text{SW} \nonumber
    \end{align}
    
\section{Experiment}
\label{thermalmodel}