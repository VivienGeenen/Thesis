\chapter{Modelling}
\label{ch:modelling}
After explaining thermal basics and the electrical analogy, the foundations are used in this chapter. How I obtain a thermal model and the resulting thermal model is introduced. Later the model is required for the MPC to predict the thermal reactions of the building. 

\section{The building}
\label{section:building}
Since the thermal model and the MPC are based on a real building, the building is described in more detail below.
\newline
The building stands at the "Campus Nord" from the "Karlsruher Institut für Technologie (KIT)" and is part of the "Energy Lab 2.0", "a research infrastructure for renewable energy"\cite{KIT.2021}. It is equipped with a kitchen, a bath room, five rooms and a technical room. For a better orientation you can see in \autoref{fig:Bauplan} a part of the construction plan of the building. The living area is around 100 $m^2$. The building has the option to use a ground-source heat pump or a air heat pump for heating and cooling. The focus is set on the air heat pump, because the most used heat pumps are air heat pumps in Germany \cite{bwp.2021}. Further information about the air heat pump you can read in the "Technische Unterlagen Montageanleiung" 

\begin{figure}
    \centering
    \def\svgwidth{320pt}
    \input{figure/Bauplan_WärmepumpenhausAuschnitt.pdf_tex}
    \caption{construction plan of the building \cite{Bauplan}}
    \label{fig:Bauplan}
\end{figure}

\section{The thermal model}
\label{thermalmodel}

\section{Data collection}

\section{Validation of the thermal Model}
\label{validationthermalmodel}



