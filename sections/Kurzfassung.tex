\chapter{Kurzfassung}
\label{ch:kurzfassung}
Die Steigerung der Einspeisung von erneuerbaren Energiequellen ins Stromnetz führt zu verstärkten Fluktuationen in der Stromerzeugung. Demand Side Management (DSM) kann genutzt werden, um diese Fluktuationen zu glätten. Dabei weisen Wärmepumpen von Wohngebäuden großes Potential auf, auf Nachfrage vom Netz genutzt zu werden. Allerdings muss der Nutzerkomfort über thermische Speicher des Gebäudes, wie Wasserspeicher oder die thermische Kapazität vom Gebäude selbst, erhalten werden. Model Predictive Control (MPC) ist ein passendes Instrument, um DSM durch die Integration von Vorhersagen zu realisieren.\newline 
Diese Thesis beinhaltet eine simulative Untersuchung einer MPC einer Wärmepumpe eines realen Gebäudes aus dem Energy Lab 2.0 am Karlsruher Institut für Technologie (KIT). Ein thermisches Gebäudemodell wird mit Hilfe von Messungen aus dem Referenzgebäude über den Grey-box- Ansatz erstellt. Um die Daten für die Grey-box Modellierung zu sammeln, wird das Gebäude mit elektrischen Heizern in zwei Experimenten beheizt, ein Experiment für den Trainingsdatensatz und ein weiteres Experiment für den Verifizierungsdatensatz.\newline
Eine MPC wird mit den Anforderungen Komfort und Netzdienlichkeit in der Kostenfunktion erstellt. Die Komfortanforderung ist abhängig von einem Belegungsplan des Gebäudes. Bei Anwesenheit von Personen wird eine komfortable Zieltemperatur vorgegeben, während bei Abwesenheit nur Netzdienlichkeit in der Kostenfunktion relevant ist. Diese MPC-Konfiguration wird als Basisszenario bezeichnet.  Weitere Szenarien werden untersucht, um den Einfluss eines Belegungsplans auf Komfort und Netzdienlichkeit zu bewerten. Eine abschließende Diskussion zeigt, dass sich der Belegungsplan bei Nutzung des Basisszenarios positiv auf den Komfortbedarf mit weniger Energieverbrauch und geringen Kosten für Netzdienlichkeit auswirkt.



