\chapter{Introduction}
\label{ch:introduction}
Climate change is challenging the entire world. In the Paris Agreement, the United Nations (UN) \nomenclature[A]{UN}{United Nations} agrees to keep the rise in global average temperature significant under two degrees Celsius \cite{UnitedNations.2015}. To achieve this aim every nation has to reduce its greenhouse gas emissions. This calls for changes in the mobility sector, industry, and energy production, for example. Germany intends to implement this by promoting electromobility, using hydrogen in industry, and energy transition \cite{Deutschlandfunk.24.06.2021}. In particular, the energy transition that has already been initiated has to be driven forward. That means the expansion of renewable energies and decreasing conventional power plants. The German government is aiming to phase out coal-fired power plants by 2038 \cite{bundesregierung.2021}. For covering the energy demand, a high increase in photovoltaics and wind power is necessary in a few years. 
\newline
Unfortunately, a disadvantage of this renewable energy is that they fluctuate with the weather and do not release energy by demand. In addition, more renewable energies lead to more intense instabilities in the grid. In the first solution approach, energy storage and demand side management (DSM) \nomenclature[A]{DSM}{Demand Side Management} are used to implement the stable grid in the future. Batteries, pumped hydroelectric energy storage, thermal energy storage, and much more could store an excess of power during a sunny or windy day. Further, DSM is clever adding and removing loads from the grid per demand and results in smoothing the grid. Load shifting is part of DSM \cite{Gellings.1985} and already used industrial. A new approach is to use residential buildings as thermal storage and demand response to contribute to grid stability. Particularly the idea of controlling the heat pumps of buildings seems promising. As at least 1.25 million heat pumps are already installed in Germany, and the tendency is increasing \cite{BMW.2021}.
\newline
The implementation of this approach needs a control strategy ensuring consumer comfort. Using the weather forecasts and prediction of the grid fluctuations improve the control. Model predictive control (MPC) \nomenclature[A]{MPC}{Model Predictive Control} is one suitable instrument to integrate forecasts and control heat pumps in buildings for stabilizing the grid with the thermal storage of the building. Research has already shown the possibilities of MPC to shift loads, to save energy and costs. This thesis goes in detail with the differences in consumption, comfort and grid-services with and without an occupancy plan of the building.

\section{Objective of this work}
\label{section:obejective}
    This thesis aims to design a control system, which simultaneously serves the grid and comply with the required comfortable internal temperature range, for the heat pump of a building in the so-called "Living Lab" of the Karlsruhe Institute of Technology (KIT) \nomenclature[A]{KIT}{Karlsruhe Institute of Technology} at Campus North. The implementation is to be carried out using the control method Model Predictive Control. This method enables to predict the future thermal behaviour of the building and to react to the actual and future fluctuations of the weather or the grid for example. 
    In the first step, a thermal model of the building behaviour must be created. For this purpose, the RC analogy is to be used. To reduce the complexity of the thermal behaviour of the building, appropriate assumptions can be made. Furthermore, the resulting model should correspond to a grey-box model, i.e., a middle ground between exact and black-box model description. After the verification of the thermal model using measured data from the Living Lab, an optimal control problem shall be created. The aim is to construct an MPC algorithm and to simulate its application. The software used will be Matlab/Simulink.

\section{Related work}
\label{section:relatedwork}

    Extant literature investigates thermal modelling and controlling of buildings. Kramer et al. \cite{Kramer.2012} summarize in a literature review thermal modelling approaches such as white-box, grey-box, and black-box models and present how researchers apply these approaches. Authors identify their thermal model parameters with measurements \cite{Siroky.2011},\cite{Hazyuk.2012}, \cite{Park.2011}, like a grey-box model or use grey-box models \cite{Freund.2020}, \cite{EvelynSperber.2019}. Coakley et al. \cite{Coakley.2014} see the advantages in the short development time for the model, fidelity of predictions, and the interaction of building, system and environmental parameters. One disadvantage is that modellers need a high level of knowledge in physical and statistical modelling \cite{Coakley.2014}.
    \newline
    Further, some authors work with such thermal models in their MPC applications for thermal management in buildings \cite{JiriCigler.}, \cite{Hazyuk.2012b}. 
    \newline
    Regardless of the type of model, MPC is utilised for control of heating, ventilation, and air conditioning (HVAC) \nomenclature[A]{HVAC}{Heating, Ventilation, and Air Conditioning} systems in buildings for a variety of reasons. Researchers are interested in the reduction of energy consumption \cite{Hazyuk.2012b} and saving costs \cite{Zwickel.2019} while obtaining thermal comfort. Some studies present how to decrease or shift the peak load of buildings \cite{Oldewurtel.2010}.
    \newline
    On the other hand, some articles refer to the potential of heat pumps for grid services.
    The report "Wärmepumpen in Bestandsgebäuden" examines, among other things, the load shifting potential of grouped heat pumps. The researchers determine 4 to 14 GWh load shifting potential for one million heat pumps \cite{WPimBestand.2020}.
    Kohlhepp and Hagenmeyer \cite{Kohlhepp.2017} also analyse the flexibility of heating systems for smart grids, partially of heat pumps. 
    \newline
    The researchers apply the above topics grey-box modelling and MPC for heat pumps to the realization of grid-services. Thus, they address thematically DSM. The paper by Avci et al. \cite{Avci.2013} gives an early indication of the potential of grid-services using real-time pricing. In the meantime, many research papers deal with prices whereby the focus in the papers differs e.g. according to the type of buildings \cite{Bianchini.2019}, \cite{Kim.2018} or the type of optimisation \cite{Bianchini.2019}, \cite{Bianchini.2016}.\newline 
    Another interesting part of research is the energy saving potential by planing the occupancy of buildings. Wang et al. show in their paper that 13 percent of energy can be saved by occupancy-based controls for an office building \cite{Wang.2019}. Researchers are trying to find out how to predict an occupancy schedule with for example machine learning approaches to better control of HVAC and consequently to save energy \cite{Liang.2016}.\newline
    
    However, is their effort necessary for an MPC with the aim of grid-services? A simple occupancy plan should be used to examine whether comfort, grid services and energy consumption can be improved compared to an MPC without an occupancy plan. Consequently, in this thesis, an MPC is created with a grey-box model and grid services are implemented with real-time pricing, similar to the papers above. However, the potential of an occupants schedule is analysed at a real reference building during focusing on the aim of grid services. This thesis finds an answer to the question: How is the difference between a view with occupants schedule and without concerning grid-services, energy consumption and comfort? 

 
%the interaction of building, system and environmental parameters vllt umbenennen, weil er es auch so nennt!!!!!!!!!!!!!!




\section{Content structuring}
\label{section:contentstructuring}
Stukturierung meiner Thesis erläutern