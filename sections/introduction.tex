\chapter{Introduction}
\label{ch:introduction}
Climate change is challenging the whole world. According to the Paris Agreement, the United Nations (UN) agrees to hold the global average temperature increase significant under two degrees Celsius \cite{UnitedNations.2015}. To reach this aim every nation is required to reduce green house gas emissions. This challenges  changes in the mobility sector, industry and energy production. Germany wants to implement this over a promotion of electromobility, using hydrogen in industry, and energy transition, called "Energiewende". Especially, the energy transition that has already been initiated must be driven forward. That means the expansion of renewable energy and decreasing conventional power plants. Some  

%Der durch den menschengemachte Klimawandel ist das Thema unserer Zeit. Im Pariser Klimaabkommen haben 929 Staaten beschlossen die Erderwärmung bis 2050 auf möglichst 2 Grad, maximal 1,5 grad zu begrenzen. Um dieses Ziel zu erreichen muss der Co2 Ausstoß in den Sektoren wie Verkehr, Industrie und deutlich reduziert werden. Ebenfalls muss die eingeleitete Klimawende weiter vorrangetrieben werden. Dies bedeutet zum einen erneuerbare Energien auszubauen und beispielsweise Energiegewinnung über klimaschädliche Braunkohle zu vermeiden. Dabei zeigen vor allem Solar und Windkraftanlagen in Deutschland die gößten Potentiale. aufgrund der temporären Abhängigkeit der Stromerzeugung resultieren im Netz Stromschwankungen. Durch das geschickte Einschalten von Energiespeichern kann das Netzwerk stabilisiert werden aufgrund Pufferwirkung. Dieser Speicher kann elektrisch, thermisch oder mechanisch ausgeführt werden. Bei den thermischen Speichern bieten sich Gebäude an, da diese in der Regel mit einem Heizsystem ausgestattet sind. Bei einem Überangebot von regnereativem Strom, aufgrund zu viel Wind bspw, kann durch geschicktes aufwärmen das Gebäude geringfügig erwärmt und das Netz so stabilisiert werden. 
%
%In Deutschland sorgen die Bemühungen der Bundesregierung zur Einhaltung des Klimaabkommens für öffentliche Diskusionen. Jüngst hat das Bundesverfassungsgericht entschieden, dass das Kimagesetz der Bundesregierung die folgenden Generation nicht ausreichend schützt.
%In Germany, there are a several   
%Kimawandel, energiewende, erneuerbare energien, netzstabilität, speichermöglichkeiten, 

\section{Objective of this work}
\label{section:obejective}
Aufgabenstellung
\section{Related work}
\label{section:relatedwork}
Bezug zu bestehenden Arbeiten
\section{Content structuring}
\label{section:contentstructuring}
Stukturierung meiner Thesis erläutern

%Your thesis should start with an introduction. The introduction is supposed to motivate your thesis.
%Discuss the relevance of your topic, why are you looking into it, why is it relevant in the field? Cite important research related to your motivation.
%Briefly state the problem as in the abstract and repeat the contribution, for example in the form of research questions. 
%
%Give an outline of your thesis.
%
%
%\begin{itemize}
%	\item Direct citation of results, an approach or similar
%	\item[] \Textcite{Fan.2015} find that their method improves the benchmark.
%	\item Indirect citation
%	\item[] Recent research highlights the importance of this method \Parencite{Fan.2015}.
%	\item Direct citation
%	\item[] \textquote{\emph{Energy optimisation in buildings is important}} \Parencite{Fan.2015}.
%\end{itemize}
