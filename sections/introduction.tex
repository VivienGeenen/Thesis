\chapter{Introduction}
\label{ch:introduction}
Climate change is challenging the entire world. In the Paris Agreement, the United Nations (UN) agrees to keep the rise in global average temperature significant under two degrees Celsius \cite{UnitedNations.2015}. To achieve this aim, every nation must reduce its greenhouse gas emissions. This calls for changes in the mobility sector, industry and energy production. Germany intends to implement this by promotion of electromobility, using hydrogen in industry, and energy transition. Especially, the energy transition that has already been initiated must be driven forward. That means the expansion of renewable energies and decreasing conventional power plants. The German government is aiming to phase out of coal-fired power plants by 2038 \cite{bundesregierung.2021}. For covering the energy demand, a high increase in photovoltaics and wind power is necessary in a few years. Since a disadvantage of this renewable energy is that they fluctuate with the weather and do not release energy by demand. In addition, more renewable energies lead to more intense instabilities in the grid. There are several possibilities to manage these problems, such as energy storage or peak shaving. Batteries, pumped hydroelectric energy storage, or thermal energy storage could store an excess of power during a sunny or windy day. On the other hand, peak shaving is clever adding and removing loads from the grid and results in smoothing the grid.
%Kimawandel, energiewende, erneuerbare energien, netzstabilität, speichermöglichkeiten, 

\section{Objective of this work}
\label{section:obejective}
Aufgabenstellung
\section{Related work}
\label{section:relatedwork}
Bezug zu bestehenden Arbeiten
\section{Content structuring}
\label{section:contentstructuring}
Stukturierung meiner Thesis erläutern

%Your thesis should start with an introduction. The introduction is supposed to motivate your thesis.
%Discuss the relevance of your topic, why are you looking into it, why is it relevant in the field? Cite important research related to your motivation.
%Briefly state the problem as in the abstract and repeat the contribution, for example in the form of research questions. 
%
%Give an outline of your thesis.
%
%
%\begin{itemize}
%	\item Direct citation of results, an approach or similar
%	\item[] \Textcite{Fan.2015} find that their method improves the benchmark.
%	\item Indirect citation
%	\item[] Recent research highlights the importance of this method \Parencite{Fan.2015}.
%	\item Direct citation
%	\item[] \textquote{\emph{Energy optimisation in buildings is important}} \Parencite{Fan.2015}.
%\end{itemize}
