\chapter{Introduction}
\label{ch:introduction}
Climate change is challenging the entire world. In the Paris Agreement, the United Nations (UN) agrees to keep the rise in global average temperature significant under two degrees Celsius \cite{UnitedNations.2015}. To achieve this aim, every nation must reduce its greenhouse gas emissions. This calls for changes in the mobility sector, industry and energy production. Germany intends to implement this by promotion of electromobility, using hydrogen in industry, and energy transition. Especially, the energy transition that has already been initiated must be driven forward. That means the expansion of renewable energies and decreasing conventional power plants. The German government is aiming to phase out of coal-fired power plants by 2038 \cite{bundesregierung.2021}. For covering the energy demand, a high increase in photovoltaics and wind power is necessary in a few years. Since a disadvantage of this renewable energy is that they fluctuate with the weather and do not release energy by demand. In addition, more renewable energies lead to more intense instabilities in the grid. Energy storage and load shifting are to be used to implement the stable grid in the future. Batteries, pumped hydroelectric energy storage, or thermal energy storage could store an excess of power during a sunny or windy day. On the other hand, load shifting is clever adding and removing loads from the grid and results in smoothing the grid. Load shifting is possible in the industry or private households. Particularly, the idea of controlling the heat pumps of buildings has great potential. As at least 1.25 million heat pumps are already installed in Germany and the tendency is increasing \cite{BMW.2021}. The thermal inertia in buildings makes them suitable for working as thermal storage. Consequently, the temporal shifting of heating in buildings in addiction with the thermal storage in buildings 

%Also Möglichkeit die zeitliche Verschiebung der Heizleistung/nutzung in kombination mit dem Gebäude als thermischen Speicher ist ideal geeignet um das Netz zu stabiliesieren. 

% Kompfort und vorhersage

\section{Objective of this work}
\label{section:obejective}
This thesis aims to design a control system, which simultaneously serves the grid and comply with the required comfortable internal temperature range, for the heat pump of a building in the so-called "Living Lab" of the Karlsruhe Institute of Technology (KIT) at Campus North. The implementation is to be carried out using the control method Model Predictive Control (MPC). This method enables to predict the future thermal behaviour of the building and to react to the actual and future fluctuations of the weather or the grid for example. 
In the first step, a thermal model of the building behaviour must be created. For this purpose, the RC analogy is to be used. To reduce the complexity of the thermal behaviour of the building, appropriate assumptions can be made. Furthermore, the resulting model should correspond to a grey-box model, i.e., a middle ground between exact and black-box model description. After validation of the thermal model using measured data from the Living Lab, an optimal control problem shall be created. The aim is to construct an MPC algorithm and to simulate its application. The software used will be Matlab/Simulink.

\section{Related work}
\label{section:relatedwork}
Bezug zu bestehenden Arbeiten
\section{Content structuring}
\label{section:contentstructuring}
Stukturierung meiner Thesis erläutern

%Your thesis should start with an introduction. The introduction is supposed to motivate your thesis.
%Discuss the relevance of your topic, why are you looking into it, why is it relevant in the field? Cite important research related to your motivation.
%Briefly state the problem as in the abstract and repeat the contribution, for example in the form of research questions. 
%
%Give an outline of your thesis.
%
%
%\begin{itemize}
%	\item Direct citation of results, an approach or similar
%	\item[] \Textcite{Fan.2015} find that their method improves the benchmark.
%	\item Indirect citation
%	\item[] Recent research highlights the importance of this method \Parencite{Fan.2015}.
%	\item Direct citation
%	\item[] \textquote{\emph{Energy optimisation in buildings is important}} \Parencite{Fan.2015}.
%\end{itemize}
