\chapter{Introduction}
\label{ch:introduction}
Climate change is challenging the entire world. In the Paris Agreement, the United Nations (UN) agrees to keep the rise in global average temperature significant under two degrees Celsius \cite{UnitedNations.2015}. To achieve this aim every nation must reduce its greenhouse gas emissions. This calls for changes e.g. in the mobility sector, industry, and energy production. Germany intends to implement this by promoting electromobility, using hydrogen in industry, and energy transition \cite{Deutschlandfunk.24.06.2021}. Especially the energy transition that has already been initiated must be driven forward. That means the expansion of renewable energies and decreasing conventional power plants. The German government is aiming to phase out coal-fired power plants by 2038 \cite{bundesregierung.2021}. For covering the energy demand, a high increase in photovoltaics and wind power is necessary in a few years. 
\newline
Since a disadvantage of this renewable energy is that they fluctuate with the weather and do not release energy by demand. In addition, more renewable energies lead to more intense instabilities in the grid. Energy storage and load shifting are to be used to implement the stable grid in the future. Batteries, pumped hydroelectric energy storage, thermal energy storage, and much more could store an excess of power during a sunny or windy day. On the other hand, load shifting is clever adding and removing loads from the grid and results in smoothing the grid. Load shifting is already used industrial. A new approach is to use residential buildings as thermal storage and demand response to contribute to grid stability. Particularly the idea of controlling the heat pumps of buildings has great potential. As at least 1.25 million heat pumps are already installed in Germany, and the tendency is increasing \cite{BMW.2021}.
\newline
The implementation of this approach needs a control strategy during consumer comfort is ensured. Using the weather forecasts and prediction of the grid fluctuations improve the control. Model predictive control (MPC) is one suitable instrument to realize this and the issue of this work. (also letzter Satz wird halb gestrichen hier folgt dann noch genauer was das Ziel ist)

\section{Objective of this work}
\label{section:obejective}
    This thesis aims to design a control system, which simultaneously serves the grid and comply with the required comfortable internal temperature range, for the heat pump of a building in the so-called "Living Lab" of the Karlsruhe Institute of Technology (KIT) at Campus North. The implementation is to be carried out using the control method Model Predictive Control (MPC). This method enables to predict the future thermal behaviour of the building and to react to the actual and future fluctuations of the weather or the grid for example. 
    In the first step, a thermal model of the building behaviour must be created. For this purpose, the RC analogy is to be used. To reduce the complexity of the thermal behaviour of the building, appropriate assumptions can be made. Furthermore, the resulting model should correspond to a grey-box model, i.e., a middle ground between exact and black-box model description. After validation of the thermal model using measured data from the Living Lab, an optimal control problem shall be created. The aim is to construct an MPC algorithm and to simulate its application. The software used will be Matlab/Simulink.

\section{Related work}
\label{section:relatedwork}

Thermal modelling and controlling of buildings are of great interest in research (weil?). Kramer et al. 
\cite{Kramer.2012}
summarizes thermal modelling approaches such as white-box, grey-box, and black-box models and presents how researchers apply these approaches. Some authors such as \cite{JiriCigler.} %in MPC paper Tabelle 11 oder so suchen (On the Selection of the Most Appropriate MPC Problem Formulation for Buildings(vllt),
use grey-box models for MPC for thermal management in buildings. 
Besonders in der MPC für temperaturregelung nutzen viele Autoren ( ...) Grey-box modelle. Autor bla untersucht blub und verwendet grey-box aus folgenden gründen.


%Zur thermischen Modellierung gibt es die Ansätze white grey and black bos modelling. Die Anwendung der verschiedenen Modellarten wurde schon getestet in blabla und in verbindung mit MPC stellt sich heraus, dass einfache, grey box Modelle sehr gut geeignet sind. 



viele Nutzen Grey-box Modelle für thermische Gebäudemodelle. Viele Nutzen MPC für Regelung von HVAC systemen. Viele wollen Energiesparen. Ich will Netz stabilisieren.

\section{Content structuring}
\label{section:contentstructuring}
Stukturierung meiner Thesis erläutern