\chapter{Introduction}
\label{ch:introduction}

\section{Motivation}
\label{section:motivation}
Hintergründe... Warum dieses Thema interessant ist
\section{Objective of this work}
\label{section:obejective}
Aufgabenstellung
\section{Related work}
\label{section:relatedwork}
Bezug zu bestehenden Arbeiten
\section{Content structuring}
\label{section:contentstructuring}
Stukturierung meiner Thesis erläutern

%Your thesis should start with an introduction. The introduction is supposed to motivate your thesis.
%Discuss the relevance of your topic, why are you looking into it, why is it relevant in the field? Cite important research related to your motivation.
%Briefly state the problem as in the abstract and repeat the contribution, for example in the form of research questions. 
%
%Give an outline of your thesis.
%
%
%\begin{itemize}
%	\item Direct citation of results, an approach or similar
%	\item[] \Textcite{Fan.2015} find that their method improves the benchmark.
%	\item Indirect citation
%	\item[] Recent research highlights the importance of this method \Parencite{Fan.2015}.
%	\item Direct citation
%	\item[] \textquote{\emph{Energy optimisation in buildings is important}} \Parencite{Fan.2015}.
%\end{itemize}
