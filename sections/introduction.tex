\chapter{Introduction}
\label{ch:introduction}
Climate change is challenging the entire world. In the Paris Agreement, the United Nations (UN) \nomenclature[A]{UN}{United Nations} agrees to keep the rise in global average temperature significant under two degrees Celsius \cite{UnitedNations.2015}. To achieve this aim every nation has to reduce its greenhouse gas emissions. This calls for changes in the mobility sector, industry, and energy production, for example. The German government intends to implement this by promoting electromobility, using hydrogen in industry, and energy transition \cite{Deutschlandfunk.24.06.2021}. In particular, the so-called energy transition that has already been initiated has to be driven forward. That means the expansion of renewable energies and decreasing conventional power plants. The German government is aiming to phase out coal-fired power plants by 2038 \cite{bundesregierung.2021}. For covering the energy demand, a high increase in energy from renewable sources, e.g. photovoltaics and wind power, is necessary for the coming years. 
\newline
Unfortunately, a disadvantage of these renewable energy sources is that they fluctuate with the weather and cannot produce energy on demand. In addition, more renewable energy sources can lead to more intense imbalances in the grid.
To compensate the imbalances in the grid, the grid requires or releases energy. In order to have enough energy available, the amount of storage options must grow with the amount of renewable energy sources. Batteries, pumped hydroelectric energy storage, thermal energy storage, and many other technologies could store an excess of power during a sunny or windy day. Furthermore, the necessity of demand side management (DSM) \nomenclature[A]{DSM}{Demand Side Management} rice also with the increasing renewable energy sources, then DSM can shift loads to balance the grid. Load shifting is part of DSM \cite{Gellings.1985} and already used in the industry. Another approach is to use residential buildings as thermal storage and demand response to contribute to grid balance\cite{Kohlhepp.2017}. As a promising DSM technology, the control of heating, ventilation, and air conditioning (HVAC) systems could be used. Particularly controlling heat pumps of buildings seem auspicious. As at least 1.25 million heat pumps are already installed in Germany, and their number is increasing steadily \cite{BMW.2021}.
\newline
The implementation of this approach needs a control strategy ensuring consumer comfort also during changing weather conditions. Model predictive control (MPC) \nomenclature[A]{MPC}{Model Predictive Control} is a suitable tool to integrate forecasts of weather and control heat pumps in buildings for stabilizing the grid with the thermal storage of the building. Research has already shown the possibilities of MPC to shift loads, to save energy and costs \cite{Oldewurtel.2010}, \cite{Hazyuk.2012b}, \cite{Zwickel.2019}.
On the other hand, researchers investigate the impact of occupancy plans on energy consumption in buildings. They prove a significant energy-saving potential \cite{Wang.2019}. This thesis picks up the advantages of an occupancy plan, and it analyses consumption, comfort, and grid service of an MPC with and without an occupancy plan of the building.
 

\section{Objective of this work}
\label{section:obejective}
    This thesis aims to design a control system, which simultaneously serves the grid and comply with the required comfortable internal temperature range, for the heat pump of a building in the so-called "Living Lab" of the Karlsruhe Institute of Technology (KIT) \nomenclature[A]{KIT}{Karlsruhe Institute of Technology} at Campus North. The implementation is to be carried out using the control method Model Predictive Control. This method enables to predict the future thermal behaviour of the building and to react to the current and future fluctuations of the weather or the grid for example. 
    In the first step, a thermal model of the building behaviour must be created. For this purpose, the physical structure of the model is to be determined. Appropriate assumptions can be made to reduce the complexity of the thermal behaviour of the building.Furthermore, the parameters of the model are determined by a parameter identification from measurements to obtain a grey-box model, i.e., a combination of physical model structure and optimisation with measurement data. After the verification of the thermal model using measured data from the Energy Lab 2.0 from the KIT, an optimal control problem shall be created. The aim is to construct an MPC algorithm and to simulate its application. The software used will be Matlab/Simulink.

\section{Related work}
\label{section:relatedwork}

    Extant literature investigates thermal modelling and controlling of buildings. Kramer et al. \cite{Kramer.2012} summarize in a literature review thermal modelling approaches such as white-box, grey-box, and black-box models and present how researchers apply these approaches. Further authors identify their thermal model parameters with measurements and use the grey-box modelling approach \cite{Harb.2016}, \cite{Freund.2020}, \cite{EvelynSperber.2019}. Coakley et al. \cite{Coakley.2014} see the advantages of grey-box modelling in the short development time for the model, fidelity of predictions, and the interaction of building, system and environmental parameters. One disadvantage is that modellers need a high level of knowledge in physical and statistical modelling \cite{Coakley.2014}. Furthermore, Cigler et al. \cite{JiriCigler.} and Hazyuk et al. \cite{Hazyuk.2012b} see the advantages of grey-box models and work with them in their MPC applications for thermal management in buildings. 
    \newline
    Regardless of the type of model, MPC is utilised for control of heating, ventilation, and air conditioning (HVAC) \nomenclature[A]{HVAC}{Heating, Ventilation, and Air Conditioning} systems in buildings for a variety of reasons, e.g. \cite{Hazyuk.2012b}, \cite{Zwickel.2019}, \cite{Oldewurtel.2010}. Researchers are interested in the reduction of energy consumption \cite{Hazyuk.2012b} and saving costs \cite{Zwickel.2019} while obtaining thermal comfort. Oldewurtel et al. \cite{Oldewurtel.2010} present how to decrease or shift the peak load of buildings.
    \newline
    On the other hand, \cite{WPimBestand.2020} and \cite{Kohlhepp.2017} refer to the potential of heat pumps for grid services. Among others, the report "Wärmepumpen in Bestandsgebäuden" examines the load shifting potential of grouped heat pumps. The researchers determine 4 to 14 GWh load shifting potential for one million heat pumps \cite{WPimBestand.2020}.
    Kohlhepp and Hagenmeyer \cite{Kohlhepp.2017} introduce a method to assess the technical potential of HVAC systems for grid services. Especially, they detect for heat pumps 5.2 TWh electrical demand in Germany per year.
    \newline
    Avci et al. \cite{Avci.2013} apply MPC for heat pumps to the realization of grid services. They give an early indication of the potential of grid services using real-time pricing. For application of DSM, most researches apply a dynamic price signal, although their focus differs e.g. according to the type of buildings \cite{Bianchini.2019}, \cite{Kim.2018} or the type of optimisation \cite{Bianchini.2019}, \cite{Bianchini.2016}.\newline 
    Another interesting field of research is the energy saving potential by planing the occupancy of buildings. Wang et al. \cite{Wang.2019} show in their paper that 13 percent of energy can be saved by occupancy-based controls for an office building. Liang et al. \cite{Liang.2016} investigate an occupancy schedule with for example machine learning approaches to better control of HVAC systems and consequently to save energy.\newline
    
    This thesis integrates occupancy schedules into the MPC formulation to investigate the potential of included occupancy behaviour on control metrics. A simple occupancy plan is used to examine whether comfort, grid services, and energy consumption can be improved compared to an MPC without an occupancy plan. Consequently, in this thesis, an MPC is created with a grey-box model and grid services are implemented with real-time pricing, similar to the papers cited above. But in this thesis, the potential of an occupants schedule is analysed at a real reference building during focusing on the aim of grid services. This thesis finds an answer to the question: How does the inclusion of an occupancy schedule in an MPC improve grid services, energy consumption and comfort? 
 %the interaction of building, system and environmental parameters vllt umbenennen, weil er es auch so nennt!!

\section{Content structuring}
\label{section:contentstructuring}
To answer the research question a thermal model is first created with help of an experiment. After that, the model can be integrated into the constructed MPC. The MPC implementation can then be used to generate the results of different scenarios.
Chapter 2 concerns foundations, which are required to solve the objective of this work. There are represented some thermal and modelling basics and an introduction in an MPC.
Chapter 3 explains the modelling approaches white-box and grey-box modelling with some advantages and disadvantages. Further, the submodels water reservoir and building model are introduced. The parameter identification is explained especially for the building model, and the verification results are shown. At last, the completed model is written in the state-space formulation.
Chapter 4 explains the structure of the experiments for the verification data set and the training data set.
Chapter 5 describes the MPC approach for a basic scenario. What kind of data, the constraints and the cost function are explained in detail. Furthermore, the workflow of the MPC is pictured, and the choice of the predictive horizon is performed. Variables are introduced concerning the comfort and grid service requirement that we use for evaluating the results.
Chapter 6 introduces the scenarios at first. After that, the results are depicted, and at last, the explanation of the results follows in the discussion.
Chapter 7 summarise this thesis and gives an outlook about future steps.