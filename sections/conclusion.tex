\chapter{Conclusion and outlook}
\label{ch:Conclusionandoutlook}

\section{Conclusion}
\label{sec:conclusion}
Increasing fluctuations due to more renewable energy sources, such as photovoltaics or wind power, make balancing the grid more difficult. More energy storage options and demand side management can remedy the situation. In this thesis, an MPC of the HVAC system of a building uses the thermal storage of the building and consumes energy with the heat pump in a favourable time for the grid. Moreover, the influence of an occupancy schedule is investigated on grid services.\newline
At first, a thermal model is generated, which predicts the future behaviour of the building's average inside temperature in the MPC. The complete model is separated into the submodels water reservoir and building model. The water reservoir is modelled with the white-box approach, while the building model is created as a grey-box model. For the latter, we identify the parameters, such as thermal resistances and capacitances, with data from the reference building and verify the building model with a verification data set. The data is generated with experiments. For that, electrical heaters are positioned in the reference building and heat the building. The data acquisition occurs via temperature sensors and sensors for recording the power consumption, which are permanently installed in the building. The collected data is used for training and verification of the building model. \newline
Using the MPC, the control signals from the heat pump and heating are optimised to regulate the indoor temperature over the predictive horizon with respect to the cost function. There, we consider constraints, which limit the control signal, dictate a temperature range or restrict states. To simulate the MPC, we use data from the reference building, which represent the weather forecasts. An occupancy schedule specifies the MPC the time when the comfort requirement is required. The dynamic price of electricity is used as an indicator for the favourable time to supply power from the grid. Furthermore, the length of the predictive horizon is determined as 24 h.\newline
At last, scenarios are investigated concerning comfort, grid service and energy consumption. The basic scenario includes the occupancy schedule and expects 22°C inside temperature during the presence of occupants. Scenario $T_\text{vary}$ respects the occupancy schedule and changes the desired temperature between 17°C and 22°C during the absence and presence of occupants. While scenario $T_\text{const}$ always desires the inside temperature at 22°C. The basic scenario is best suited to combine comfort, energy consumption and grid services.\newline

\section{Outlook}
\label{sec:outlook}
In future, the heating system of the reference building will be readily installed, and sensors will measure the heat flow from the heating in the building. This enables a new estimation of the grey-box model with data from an ordinary heat period. To generate the new data, no artificial heating is necessary in an experiment, instead the recorded data of habitual office days can be used. The intention is to create a more representative model.\newline
Furthermore, a comparison of the advantages and disadvantages between an MPC with one model for winter and summer and an MPC with different models for winter and summer is engaging. Therefore, an MPC with two models can be prepared and compared with the approach in this thesis.\newline
A further interesting issue is the investigation of the advancement of the MPC for the use of heat pumps in buildings as a control reserve of the grid. Günther et al. \cite{WPimBestand.2020} already investigate the integration of heat pumps into the grid, but without an MPC approach. 


%A building without its own power generation can only draw negative power from the grid. Therefore, only the secondary and tertiary control reserves are of interest, which means that according to the rules for control reserves, a connection is only possible with negative power.

%Repeat the problem and its relevance, as well as the contribution (plus quantitative results). 
%Look back at what you have written in the introduction.
%
%Provide an outlook for further research steps.